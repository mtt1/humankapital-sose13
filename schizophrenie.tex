% preamble {{{
\documentclass[12pt,
               DIV13,
               paper=a4,
               twoside=false,
               onehalfspacing,
               %titlepage,
               bibliography=totoc,
               toc=graduated,
               draft,
               ]{scrartcl}

\usepackage[utf8]{inputenc}
\usepackage[T1]{fontenc}
%\usepackage{ngerman}
%\usepackage[british]{babel}
\usepackage[ngerman]{babel}
\usepackage[babel,german=quotes]{csquotes}
\usepackage{setspace}
%\usepackage{mathptmx}           % pslatex's successor
%\usepackage[scaled=.92]{helvet} % pslatex's successor
%\usepackage{courier}            % pslatex's successor
\usepackage[osf]{libertine}
\usepackage{courier}
% GEHT NICHT?:
%\usefont{T1}{fxlj}{m}{n}\selectfont % Mit Zahlen, die nach unten hängen
\usepackage{color}
\usepackage{ifpdf}
\usepackage{scrpage2}
\usepackage{xspace}
\usepackage[babel=true]{microtype}

\usepackage{amssymb,amsmath}

\usepackage[backend=biber,
            sortlocale=de,
            %style=authoryear,
            %citestyle=authoryear-ibid,
            citestyle=authoryear-icomp,
            bibstyle=authoryear,
            %natbib=true,
            sortlos=los,
            %autopunct=true,
            language=ngerman,
            clearlang=true,
            %babel=none,
            block=none,
            %ibidtracker=constrict, % automatically set by authoryear-icomp
            loccittracker=constrict, % no page-number after "ibidem" if the same page is cited again
            ]{biblatex}
\addbibresource{literatur.bib}

\ifpdf
 \usepackage[pdftex]{graphicx}
 \DeclareGraphicsExtensions{.pdf}
 \pdfcompresslevel=9
% \usepackage[%
%   pdftex=true,
%   backref=true,
%   linktocpage=true,
%   pdfpagemode=None
% ]{hyperref}
% \hypersetup{
%   pdftitle={},
%   pdfauthor={Matthias Rudolph},
%   pdfsubject={},
%   pdfcreator={LaTeX2e and pdfLaTeX},
%   pdfproducer={},
%   pdfkeywords={}
% }
\else
  \usepackage[dvips]{graphicx}
  \DeclareGraphicsExtensions{.eps}
\fi

%\setcounter{secnumdepth}{-1} % keine section-Nummerierung

\pagestyle{scrheadings}
%\automark[section]{subsection}

\ihead[]{}
\chead[]{Connect -- I -- Cut} % Name der Hausarbeit
\ohead[]{Matthias Rudolph}
\ifoot[]{}
\cfoot[]{}
\ofoot[]{\thepage} % Seitenzahl

% Biblatex hacks
%
% Quick 'n' dirty
% Richtiger Artikel für "Hrsg. *vom* Institut"
\DefineBibliographyStrings{ngerman}{%
    bytranslator = {hrsg\adddotspace vom},
}

% Kein "Bd." in \volcite (trotzdem per Komma getrennt)
%\DeclareFieldFormat{volcitevolume}{#1}

% Doppelpunkt statt Punkt nach dem Label
\renewcommand{\labelnamepunct}{\addcolon\space}

% Schrägstriche zwischen mehreren AutorInnen
\renewcommand*{\multinamedelim}{\addslash}
\renewcommand*{\finalnamedelim}{\addslash}

% Alle Bibliografie-Einträge: Nachname, Vorname. Auch für Einträge
% mit mehreren AutorInnen und die Sigel-Liste.
% http://tex.stackexchange.com/questions/12806/guidelines-for-customizing-biblatex-styles
\DeclareNameAlias{sortname}{last-first} % Bibliografie
\DeclareNameAlias{default}{last-first} % Vollzitate (?)
%\DeclareNameAlias{labelname}{last-first} % alle anderen Zitate

% Nachname in Kapitälchen. Effekt aber nicht nur in der Bibliographie,
% sondern auch bei Zitaten.
%\renewcommand*{\mkbibnamelast}[1]{\textsc{#1}}

% More space between different entries in the bibliography.
% See \bibitemsep, \bibnamesep, \bibinitsep
% http://tex.stackexchange.com/questions/19105/how-can-i-put-more-space-between-bibliography-entries-biblatex
\setlength\bibnamesep{1em}

\newcommand{\freel}{\vspace{1em}}
\newcommand{\lips}{\dots\unkern}

\newcommand{\tit}[1]{\textit{#1}}
\newcommand{\tbf}[1]{\textbf{#1}}

\newcommand{\pc}[2]{\parencite[#1]{#2}}
\newcommand{\vgl}[2]{\parencite[vgl.][#1]{#2}}
\newcommand{\zn}[3]{\parencite[#1, zit. nach][#2]{#3}}

% Disable single lines at the start of a paragraph (Schusterjungen)
\clubpenalty = 10000

% Disable single lines at the end of a paragraph (Hurenkinder)
\widowpenalty = 10000
\displaywidowpenalty = 10000

% Don't insert more space at the end of a sentence than between words.
\frenchspacing

% Worries in Blau
\usepackage{ifdraft}
\newcommand{\worries}[1]{\ifdraft{\textcolor{blue}{\texttt{(#1)}}}{}}

% }}}

% Für diese Arbeit {{{

% Fürs Kapital
\newcommand{\gwg}{G--W--G'\xspace}
\newcommand{\wgw}{W--G--W\xspace}
\newcommand{\cic}{Connect -- I -- cut\xspace}

\newcommand{\dg}{Deleuze und Guattari\xspace}

\newcommand{\decod}{Decodierung\xspace}
\newcommand{\deterr}{Deterritorialisierung\xspace}

\newcommand{\hoe}{\tit{Homo oeconomicus}\xspace}

\newcommand{\gwapm}{G--W$<^{\text{A}}_{\text{Pm}}$\xspace}
\newcommand{\gwapmp}{G--W$<^{\text{A}}_{\text{Pm}}$ \dots P\xspace}
%\newcommand{\gwapm}{G--W$<\genfrac{}{}{0pt}{}{A}{Pm}$\xspace}

% Footnote tricks
% default:
%\deffootnote[1em]{1.5em}{1em}{%
%    \textsuperscript{\thefootnotemark}}

%\deffootnote[1em]{1em}{1em}{%
%    \textsuperscript{\thefootnotemark}}

%\deffootnote[1em]{1.5em}{1em}{\thefootnotemark\enspace}

%\deffootnote{1em}{1em}{\thefootnotemark\enspace}

% }}}

% Titel {{{

\begin{document}
\setcounter{page}{0}

\titlehead{Goethe-Universität Frankfurt am Main\\
Fachbereich Philosophie und Geschichtswissenschaften\\
Institut für Philosophie\\
Prof. Dr. Christoph Menke\\
Seminar: Demokratie und Kapitalismus\\
SoSe 2013\\
Modul: VM 3b}
%\title{Das Subjekt des Humankapitals}
%\title{Die Selbstverwertung des Selbst/Subjekts}
\title{Connect -- I -- cut}
%\subtitle{Zwischen Steigerungslogik und Schizophrenie}
\subtitle{Die schizophrene Struktur der Logik des Kapitals}
\author{Matthias Rudolph}
\date{Vorgelegt am: \today}

\maketitle
\vfill

\noindent Matthias Rudolph\\
Frankenallee 117\\
60326 Frankfurt/M\\
Matr.-Nr.: 5273120\\
Mod. Mag. Philosophie (7. FS), NF Soziologie (3. FS) \& Politikwissenschaft (2. FS)\\ % anpassen
mttrud@gmail.com
\newpage

\tableofcontents
\newpage

%}}}

% Einleitung {{{
\section{Einleitung}

Kombination zweier Thesen:

Krankheit unserer Zeit (Adorno und Deleuze)

Innere Komposition ableiten (Adorno)/Mechanismen (Deleuze)

% }}}

\section{Hauptteil}

%\subsection{Schizophrenie und Kapitalismus}

% {{{ Schizo
%\subsubsection{Schizophrenie}
\subsection{Schizophrenie}

\worries{Es geht nicht um Pathogenese, nicht um Medizin, nicht um
Therapie, nicht um Entstehung, Ursachen, Wirkungen usw. Ganz
selektiver Blick in dieser Rekonstruktion}

Die Diagnose der Schizophrenie, die Deleuze und Guattari formulieren,
ist nicht einfach in der landläufigen Bedeutung als "`gespaltene
Persönlichkeit"' zu übersetzen. Hinter dem Begriff der Schizophrenie
steckt auch bei Deleuze und Guattari ein medizinisch, psychologisch
und philosophisch komplexes Konzept.

In der Fachliteratur gilt Schizophrenie als Krankheit ohne
einheitliches Krankheitsbild: "`\emph{Die} Schizophrenie als homogenes
Krankheitsbild mit einheitlichem klinischem Erscheinungsbild und einem
eindeutig vorhersagbaren Krankheitsverlauf mit immer wieder
vergleichbaren ähnlichen Krankheitsstadien gibt es nicht"'
\pc{S. 799, Hervorh. im Orig.}{psych}.

Dennoch lässt sich eine gewisse Bandbreite schizophrenen Verhaltens
beschreiben: Die "`beiden Pole der Schizophrenie"' \pc{21}{schizg}
sind die katatonischen Krampfzustände und die wahnhafte Aktivität
oder, um es direkt in Deleuze'schen Begriffen zu sagen: "`Katatonie
des organlosen Körpers, anorganische Tätigkeit der Organmaschinen"'
\pc{21}{schizg}. Also auf der einen Seite eine "`Zersetzung der
Person"' und eine "`Abspaltung von der Realität, die einem starren und
sich selbst verschlossenen Innenleben eine Art Übergewicht oder
Autonomie verleihen"' \pc{23}{schizg} -- das, was üblicherweise
Autismus genannt wird \vgl{801}{psych}. Auf der anderen die
"`Zersplitterung oder funktionale Dislokation der Assoziationen"'
\pc{23}{schizg}, eine Betrachtung, "`die den fehlenden Zusammenhang
zur Hauptstörung macht"' \pc{23}{schizg}.

\worries{\gwg auch als Autismus?}

%Im folgenden wird vor allem die Seite des fehlenden Zusammenhangs, der
%Diskontinuität von Bedeutung sein.
%
%--

Anders als vielen gängigen Interpretationen geht es Deleuze und
Guattari aber darum, die Schizophrenie nicht im Sinne eines Mangels (etwa
von Kontinuität, Konsistenz, Sinn, Signifikant, Vater o.\,ä.
\vgl{24}{schizg} \worries{+ ganz woanders?}) zu interpretieren, sie
nicht auf die "`Merkmale des Defizits"', der "` Zerstörung"', der
"`Lücken und Spaltungen"' \pc{24}{schizg} zu reduzieren, die für den
fehlenden Zusammenhang verantwortlich gemacht werden. Stattdessen
"`besteht die Schwierigkeit darin, der Schizophrenie in ihrer
Positivität und als Positivität Rechnung zu tragen"' \pc{24}{schizg}.

Schizophrenie wird also nicht vor allem als Pathologie gedeutet, die
es zu vermeiden oder zu heilen gelte, nicht so sehr als Problem oder
Sackgasse, sondern vielleicht sogar als etwas mit Potenzial, so
kontraintuitiv das auf den ersten Blick auch wirken mag. Potenzial
dafür, allem Festen zu entkomen, den Codes, den Territorialitäten, den
Axiomen zu entgehen: "`Man kann sagen, daß der Schizophrene [\lips] \emph{alle Codes
durcheinanderbringt}"' \pc{S. 22, Hervorh. im Orig.}{ao}.

"`Das schizoide Werk par excellence"' \pc{54}{ao} sind "`Puzzleteile,
die aber nicht zu einem, sondern verschiedenen Puzzles gehören:
[\lips] mit ihren nicht zueinander passenden Rändern, die gewaltsam
ineinandergezwängt, ineinandergeschachtelt werden und stets Reste
übrig lassen"' \pc{54}{ao}.

Die Schizophrenie ist aber nicht als kontinuierliche Flucht zu
verstehen, als stetige Bewegung des Unterlaufens oder Entwischens. \dg
bezeichen Schizophrenie stattdessen in einem ganz bestimmten Sinne als
Prozess: Prozess als "`Bruch, Einbruch, Durchbruch, der die Kontinuität einer
Persönlichkeit unterbricht und sie auf eine Art Reise schickt, durch
ein intensives und erschreckendes \glq Mehr an Realität\grq{}
hindurch, gemäß Fluchtlinien, in denen Natur und Geschichte,
Organismus und Geist sich verfangen"' \pc{28}{schizg}. Prozess ist
also kein stetiges Fließen, kein kontinuierlicher Übergang etwa
zwischen verschiedenen Stadien der Schizophrenie.

Dies betont erneut die diskontinuierliche Seite der Schizophrenie,
also die Seite der wahnhaften Aktivität, des fehlenden Zusammenhangs.
Dieses Verständnis bleibt allerdings auch wiederum nicht bei der
Betrachtung voneinander getrennter Einheiten stehen. Schizophrenie ist
deshalb auch nicht gleichbedeutend mit gespaltener Persönlichkeit.
"`Spaltung ist ein schlechtes Wort zur Bezeichnung des Zustands"'
\pc{27}{schizg}, weil es nicht um zwei separate Einheiten geht,
sondern um den "`Bruch, Einbruch, Durchbruch"' \pc{28}{schizg}, die
eben in der Unterbrechung doch verbinden. So ist die Schizophrenie die
"`Herstellung einer nicht-lokalisierbaren Verbindung"' \pc{19}{schizg}
zwischen heterogenen Elementen, so dass sie, "`\emph{gerade weil sie
keine Beziehung zueinander haben}, untereinander in Beziehung treten"'
\pc{19}{schizg}.

Auf diese Weise charakterisiert, werde deutlich, dass die
Schizophrenie die Krankheit unserer Zeit sei: "`Wenn die Schizophrenie
als die Krankheit der heutigen Zeit erscheint, dann nicht aufgrund von
Allgemeinheiten, die unsere Lebensweise betreffen, sondern in bezug
auf äußerst präzise Mechanismen ökonomischer, sozialer und politischer
Natur"' \pc{28}{schizg}. Diesen Mechanismen gilt es im Folgenden
nachzuspüren.

% }}}

% und Kap.mus {{{

%\subsubsection{\dots und Kapitalismus}
\subsection{\dots und Kapitalismus}

"`In der Tat meinen wir, daß der Kapitalismus im Zuge seines
Produktionsprozesses eine ungeheure schizophrene Ladung erzeugt, auf
der wohl seine Repression lastet, die sich aber unaufhörlich als
Grenze des Prozesses reproduziert"' \pc{45}{ao}, so beschreiben
Deleuze und Guattari den Zusammenhang von Kapitalismus und
Schizophrenie.

Dieser Zusammenhang lässt sich nun auf zwei Arten untersuchen. Zum
einen gibt es die Ebene der Dynamiken, die vom Kapitalismus erzeugt
werden, und die das erzeugen, was vielleicht dem an nahesten kommt,
was Deleuze und Guattari oben "`schizophrene Ladung"' nennen. Es
handelt sich hier um die Bewegungen der Decodierung und
Deterritorialisierung, um Veränderungen auf gesellschaftlicher Ebene,
um Mechanismen der Regierung usw.

(Das genügt aber \dg nicht ... )

Noch fruchtbarer scheint es mir aber zu sein, das Argument des
Zusammenhangs von Schizophrenie und Kapitalismus noch auf einer
anderen Ebene zu wiederholen, nämlich auf formaler Ebene, der Ebene
der Logik des Kapitals, d.\,h. der Form(el) \gwg.

Im Folgenden werde ich zunächst kurz die ... rekonstruieren/anreißen,
um dann auf die Formel \gwg zurückzukommen und zu zeigen, inwiefern
die grundlegende Struktur des Kapitalismus bereits eine schizophrene
Struktur aufweist.

%Die spezifischen Veränderungen, die sich mit dem Aufkommen des
%Kapitals ergeben, bezeichnen sie dabei als Decodierung und
%Deterritorialisierung -- zwei Begriffe, die im Kontext eines
%besonderen Vokabulars der Maschinen usw. \worries{?} stehen.

Die Beschreibungen des Kapitalismus als schizophrenes System stehen im
Kontext einer besonderen (beinahe geschichtsphilosophisch anmutenden)
Konzeption der Abfolge von Gesellschaften. Deleuze und Guattari
unterscheiden territoriale, despotische und kapitalistische
Gesellschaften bzw. Gesellschaftsmaschinen \vgl{338}{ao}.

Frühere Gesellschaften seien gekennzeichnet gewesen von klaren
Territorialitäten (Ländern, Zünften, Ethnien usw.) und Codes (z.\,B.
der Zusammenhang von Konsumgütern und Prestige und die Übersetzung des
einen ins andere) \vgl{318, 332}{ao} im Falle der territorialen
Gesellschaften, bzw. Übercodierung \worries{BSP} in despotischen
Gesellschaften.

Das Besondere am Kapitalismus sei im Gegensatz zu allen früheren
Gesellschaften die allgemeine Decodierung und Deterritorialisierung
\vgl{337}{ao}.

"`Denn jedes Fließen des Stroms ist
Deterritorialisierung, jede verschobene Grenze Decodierung"'
\pc{298}{ao}.

Diese Begrifflichkeiten lassen sich zur Verdeutlichung auch auf Marx'
klassische Analyse der Entstehung des Kapitalismus im England des
\worries{xx.} Jahrhunderts anwenden. Was Marx "`ursprüngliche
Akkumulation"' nennt, erscheint bei Deleuze und Guattari als
nachhaltige Bewegung der Decodierung und Deterritorialisierung.

Kurz gesagt, war es eine Bedingung für das Entstehen des Kapitalismus,
dass sich doppelt-freie Lohnarbeiter*innen und Geld, das zu ihrer
Bezahlung ausgegeben werden konnte, begegnet sind \vgl{xx}{kap}. In
Begriffen der Decodierung und Deterritorialisierung wird das gefasst
als: "`Tatsächlich entsteht er [der Kapitalismus] aus dem
Zusammentreffen zweier Arten von Strömen: den decodierten
Produktionsströmen in Form des Geld-Kapitals und den decodierten
Arbeitsströmen in Form des \glq freien Arbeiters\grq\,"' \pc{44}{ao}.

Nach der Deterritorialisierung der Despotenmaschine, die die Ströme
aber gewissermaßen wieder einfängt und übercodiert, vollzieht die
Kapitalmaschine eine Bewegung, die "`nichts von den Codes und
Übercodierungen übrigläßt"' \pc{337}{ao}.

Ihr Schlüssel ist dagegen die Axiomatik, die für alle Eventualitäten
und Entwicklungen beständig neue Axiome produziert.

\worries{Aber immer wieder Axiomatik, neue Axiome}

\worries{Mehr: Organloser voller Körper usw.}

--

Aber auch \dg bleiben nicht bei den Verschiebungen/Oberfläche stehen
...

"`Wir haben gesehen, daß das Verhältnis von
Kapitalismus und Schizophrenie bei weitem über die Probleme der
Lebensweise, der Umwelt, der Ideologie usw. hinausgeht, und auf der
grundlegenden Ebene ein und derselben Ökonomie, ein und desselben
Produktionsprozesses gesehen werden muß. Unsere Gesellschaft
produziert Schizos wie Haarwaschmittel oder wie VWs mit dem einzigen
Unterschied, daß jene nicht ver-|käuflich sind"' \pc{S. 315 f.}{ao}.

Wenn das Verhältnis von Kapitalismus und Schizophrenie auf "`ein und
derselben Ökonomie"' beruht, dann muss es sich dort, in der Analyse
der Ökonomie, auch wiederfinden lassen. Und das ist es, was ich im
Folgenden zu zeigen versuche: die schizophrene Struktur in der
Kapital-Logik selbst.

Dafür zunächst aber einmal die Marxsche Analyse der Logik des Kapitals
rekonstruieren, um die Punkte herauszuarbeiten, die für die zu
stellende Diagnose relevant sind.

---

"`Im Differentialquotienten ist die grundlegende kapitalistische
Erscheinung zum Ausdruck gebracht: \emph{die Transformation des
Mehrwerts an Code in Mehrwert an Strömen}"' \pc{S. 292 f.}{ao}.

---

--> Lenger, das Erste ist das Dritte

"`Ist es in diesem Sinne richtig zu sagen, daß die Schizophrenie das
Produkt der kapitalistischen Maschine sei, wie die depressive Manie
und die Paranoia Produkt der Despotenmaschine und die Hysterie Produkt
der Territorialmaschine?"' \pc{44}{ao}.

"`Die Tendenz besitzt einzig eine interne Grenze [aber keine äußere],
die sie überschreitet, allerdings indem sie sie verschiebt, das heißt
sie rekonstituiert, sie als interne Grenze, die erneut mittels
Verschiebung überschritten \worries{Selbst-Überschreitung} werden muß,
wiederfindet: so erzeugt sich die Kontinuität des kapitalistischen
Prozesses in diesem stets verschobenen Einschnitt des Einschnitts
(coupure de coupure), anders gesagt in der \emph{Einheit von Spaltung
(schize) und Strom}"' \pc{S. 296, meine Hervorh.}{ao}.

% }}}

% GWG {{{

\subsection{\gwg}
\label{gwg}

\gwg lautet die vielbeschworene Formel des Kapitals. Sie drückt aus,
wie auf dem Umweg über eine bestimmte Ware (W) aus Geld (G) mehr Geld
(G') wird. Nach Marx' Theorie ist unter Kapital nicht etwa einfach
Geld oder eine Ansammlungen von Maschinen in einer Fabrik zu
verstehen, sondern eine Wertsumme. Allerdings nicht einfach irgendeine
Wertsumme, sondern der Wert \emph{in Bewegung}, in eben der Bewegung,
die durch die Formel \gwg beschrieben wird, in der der Wert also
entweder in der Form des Geldes oder in der der Ware auftritt.

%-- GWG vs WGW

%-- WGW

Wir befinden uns an dieser Stelle in der sogenannten
Zirkulationssphäre, in der Privateigentümer*innen Waren und Geld als
Äquivalente tauschen. Von der einen Seite stellt sich dieser Tausch
als Verwandlung von Ware in Geld (W--G), von der anderen Seite als
Verwandlung von Geld in Ware (G--W) dar. Diese Bausteine lassen sich
auf zwei verschiedene Weisen verketten, zum einen als \wgw, zum
anderen umgekehrt als G--W--G. Der erste Fall ließe sich als
"`Verkaufen, um zu kaufen"' beschreiben, der zweite als "`Kaufen, um
zu verkaufen"' \vgl{162}{kap}. Der entscheidene Unterschied liegt in
der damit ausgesprochenen Zielsetzung: Das Ziel von \wgw ist der
Austausch einer Ware gegen Geld, um damit eine andere Ware zu kaufen.
%"`Das Geld ist also definitiv ausgegeben"' \pc{163}{kap}.
Die Befriedigung eines Bedürfnisses mit der zweiten Ware, also ihr
Entzug aus der Zirkulation durch den Konsum, ist das Ziel. So besitzt
der Kreislauf \wgw einen ihm äußerlichen Zweck, und damit seine Grenze
und sein Ende. "`Konsumtion, Befriedigung von Bedürfnissen, mit einem
Wort, Gebrauchswert ist daher sein Endzweck"' \pc{164}{kap}.

%-- GWG

Anders im Falle von G--W--G. Ziel ist hier nicht der Entzug einer Ware
aus der Zirkulation, und ebensowenig der Entzug des Geldes. Es wird
vielmehr ausgegeben, in die Zirkulation hineingeworfen, damit es am
Ende zurückkommt. Das Geld ist "`nur vorgeschossen"' \pc{163}{kap}.
Nun wäre die ganze Bewegung "`eine ebenso zwecklose als abgeschmackte
Operation"' \pc{165}{kap} und die Mühe sinnlos, stünde am Ende als
Resultat nur "`tautologisch"' \pc{164}{kap} das, womit begonnen wurde,
nachdem es obendrein einmal dem Risiko des Verlusts ausgesetzt war.
Der bloße Erhalt des Werts kann also nicht das Ziel sein. Ebensowenig
geht es aber um Erreichung eines vom Anfang qualitativ verschiedenen
Endpunkts. Denn: "`Eine Geldsumme kann sich von der andren Geldsumme
überhaupt nur durch ihre Größe unterscheiden. Der Prozeß G--W--G
schuldet seinen Inhalt daher keinem qualitativen Unterschied seiner
Extreme, denn sie sind beide Geld, sondern nur ihrer quantitativen
Verschiedenheit"' \pc{165}{kap}. Der Inhalt dieser Bewegung besteht
also darin, dass an ihrem Ende \emph{mehr} Geld steht als am Anfang.
An die Stelle des qualitativen Unterschieds der beiden Waren in \wgw
tritt hier ein quantitativer Unterschied der beiden Geldsummen. Aus
der zweiten Art, die Bausteine G--W und W--G zusammenzusetzen, wird
also \gwg, mit G'$>$G.

%\footnote{Das Rätsel, woher diese Differenz, d.\,h. diese
%Vergrößerung der Wertsumme kommt, löst Marx durch die Einführung der
%Differenz von Arbeit und Arbeitskraft und die Verschiebung des
%Blickwinkels von der Zirkulation auf die Produktion. Die Ware
%Arbeitskraft kann in der Zirkulation gekauft werden. Ihr spezifischer
%Gebrauchswert ist aber Arbeit, also wertbildende Tätigkeit, die im
%Produktionsprozess verausgabt wird \vgl{xx}{kap}. Die Betrachtung der
%besonderen Ware Arbeitskraft, wie der ganzen Sphäre der Produktion,
%lasse ich an dieser Stelle außen vor. Für die Bestimmung der
%Steigerungslogik des Kapitals ist es unerheblich, woher die Steigerung
%kommt, solange sie nur nicht als zufällige Nebenerscheinung verstanden
%wird.\worries{?}}

%-- Steigerung / endlos / maßlos

Hier "`sind Anfang und Ende dasselbe, Geld, Tauschwert, und schon
dadurch ist die Bewegung endlos"' \pc{166}{kap}. Es handelt sich also
nicht um einen einfachen Kreislauf, es kreist nicht immer
unterschiedslos die gleiche Wertsumme, sondern um eine Bewegung, die
jeweils auf einer höheren Stufe wiederholt wird. "`Das Ende jedes
einzelnen Kreislaufs [\lips] bildet daher von selbst den Anfang eines
neuen Kreislaufs"' \pc{S. 166 f.}{kap}. Und jedes Mal steht am Ende
eine Vermehrung der eingesetzten Wertsumme. Zum Ziel dieser endlos
wiederholten Operation wird einzig die Vermehrung des Geldes, aber
eben nicht sein Entzug aus der Zirkulation. Würde die Bewegung
angehalten, "`hörten [die beispielhaften 100 Pfd. St.] auf, Kapital zu
sein. Der Zirkulation entzogen, versteinern sie zum Schatz"'
\pc{166}{kap}.

"`Die Zirkulation des Geldes als Kapital ist dagegen Selbstzweck, denn
die Verwertung des Werts existiert nur innerhalb dieser stets
erneuerten Bewegung"' \pc{167}{kap}.

%Dass am Ende eines Zyklus aus Geld mehr Geld geworden ist, bestimmt
%Marx als die spezifische Steigerungslogik des Kapitals. Dabei geht es
%nicht um die Einzelfälle, in denen geschickte Händlerinnen ungeschickte
%Käuferinnen übers Ohr hauen, sondern die Wertzunahme, die Produktion
%von Mehrwert, ohne die das Kapital kein Kapital ist.

%== Zu den Besonderheiten/Widersprüchlichkeiten dieser Bestimmung

%-- Form-Wechsel / Keine Substanz

Neben dieser spezifischen Steigerungslogik des Kapitals ist an dieser
Stelle aber insbesondere die Form der Bewegung von Interesse. Kapital
ist nicht einfach Geld, was aus sich selbst mehr Geld macht, wie im
"`Lapidarstil"' \pc{170}{kap} das zinstragende Kapital vorgestellt ist
und was zugegebenermaßen schon eine bemerkenswerte Eigenschaft wäre.
In seiner allgemeinen Form ist Kapital Geld, das über den Umweg seiner
Verausgabung als mehr Geld zu sich selbst zurückkommt. Es handelt sich
also nicht nur um eine Vermehrung, eine Steigerung, sondern obendrein
um eine Selbst-Verausgabung und eine Identität durch diese
Verausgabung hindurch. Das Kapital ist Kapital durch einen zweifachen
Wechsel der Form, einmal von Geld zu Ware und einmal von Ware zurück
zu Geld: "`beide, Ware und Geld, [funktionieren] nur als verschiedne
Existenzweisen des Werts selbst [\lips] Er geht beständig aus der
einen Form in die andre über, ohne sich in dieser Bewegung zu
verlieren"' \pc{S. 168 f.}{kap}.

%-- Kapital als Subjekt

Ein Satz von Helmut Reichelt müsste an dieser Stelle reformuliert
werden. Er sagte: "`am Ende der Darstellung wird sich zeigen, daß es
das Kapital selbst ist, das uns in verschiedenen Formen begegnet, die
sich alle als Momente seiner selbst erweisen"' \pc{181}{reichelt}.
Passender noch wäre es zu sagen, "`daß es das Kapital selbst ist, das
\emph{sich} in verschiedenen Formen begegnet"'. Selbst angetrieben,
die ständigen Verwandlungen überstehend, wird das Kapital zum
"`übergreifende[n] Subjekt"' \pc{169}{kap} seiner eigenen Verwertung.
Das Kapital ist das "`Subjekt der beschriebenen Bewegung, die es
selbst als sein eigner Verwertungsprozeß ist"'
\zn{Marx}{181}{reichelt}. In der Bewegung der Selbstverwertung nennt
Marx das Kapital schließlich "`automatisches Subjekt"' \pc{169}{kap},
es ist "`selbst der prozessierende Widerspruch"' \pc{601}{grundr}.

Schlusssätze ...

---

%"`, und verwandelt sich so in ein automatisches Subjekt"' \pc{S. 168 f.}{kap}.

%\worries{vs. S. 169, wo Marx den Wert "`Substanz"' nennt}

%Das Kapital "`ist immer sich selbst voraus, stets entgeht ein Rest der
%Zuschreibung"' \pc{125}{strauss}. Selbstüberschreitung

%"`In der Tat aber wird der Wert hier das Subjekt eines Prozesses,
%worin er unter dem beständigen Wechsel der Formen von Geld und Ware
%seine Größe selbst verändert, sich als Mehrwert von sich selbst als
%ursprünglichem Wert abstößt, sich selbst verwertet. Denn die Bewegung,
%worin er Mehrwert zusetzt, ist seine eigne Bewegung, seine Verwertung
%also Selbstverwertung"' \pc{169}{kap}.

%-- Widerspruch / Selbst-Negation

Am bemerkenswertesten ist aber eine Formulierung, die Helmut Reichelt
zitiert: "`Das Kapital ist daher in jeder besonderen Phase die
Negation seiner als des Subjekts der verschiednen Wandlungen"'
\zn{Marx}{181}{reichelt}. Diese eigentümliche Bewegung des Kapitals,
die über einen Umweg zu sich selbst zurückkommt, wird hier verstanden
als Negation.

Die Identität des Kapitals lässt sich nur noch bestimmen als
substanzlose Identität in der Differenz der Steigerung und der
zweifachen Verwandlung, ja der Selbst-Negation. \worries{?}

% das sich selbstnegierende Subjekt

% }}}

% Connect-I-cut {{{

\subsection{Die Schizophrenie der Logik (?): \cic}

%\worries{Aber was hat das alles mit Schizophrenie zu tun?}
%
%\worries{Einmal allgemein Connect-I-cut und warum das schizophren ist.
%Dann: keine einfache Parallelität. Dann die drei Unterkapitel explizit
%für \gwg}

Beinahe nebenbei werfen \dg ein kleines Wortspiel ein, das aber von
besonderer Anschaulichkeit ist. Durch Einfügung von zwei Bindestrichen
wird aus "`Connecticut"' "`\cic"' \pc{48}{ao}. Dieses Wortspiel soll
mir im Folgenden als Modell dienen, wobei ich es von seinen
Territorialitäten lösen möchte: zum einen dem US-Bundesstaat, zum
anderen von seinem Ursprung, dem Bericht des Psychoanalytikers Bruno
Bettelheim über die Behandlung des autistischen Kindes Joey
\parencites[vgl.][]{joey}[306-446]{emptyf}.

Der Fall Joey ist aufgrund seiner Eigentümlichkeit für \dg ohnehin von
besonderem Interesse, denn Joey wird auch \enquote{a
\enquote{mechanical boy}} genannt \pc{3}{joey}. Joey erlebt sich
selbst als Maschine oder zumindest als von Maschinen gesteuert, kann
mit der Umwelt nur auf vorgestellte maschinelle Weise interagieren,
durch imaginäre Röhren, Kabel, Drähte usw. \vgl{3}{joey}. Ohne dies
bleibt er abgeschnitten von der Umwelt, in sich zurückgezogen, ja
schlicht ohne Möglichkeit der Interaktion, da die  Organe, die
Maschinen fehlen. Ein paradigmatischer Fall der beiden Pole der
Schizophrenie, wie \dg sie fassen: Aktivität der Organmaschinen und
Katatonie des organlosen Körpers. "`For long periods of time, when his
\glq machinery\grq{} was idle, we would sit so quietly that he would
disappear from the focus of the most conscientious observation. Yet in
the next moment he might be \glq working\grq{} and the center of our
captivated attention"' \pc{3}{joey}.

Aber nicht nur die beiden Pole kommen deutlich zur Geltung, der ganze
Kontext maschineller Begrifflichkeiten spielt im Falle Joeys eine
zentrale Rolle: "`To do justice to Joey I would have to compare
simultaneously to a most inept infant and a highly complex piece of
machinery"' \pc{3}{joey}.

An einem bestimmten Punkt der Therapie \vgl{398-404}{emptyf}
beschreibt sich Joey eine zeitlang als Indianer, als
"`Connecticut-Indianer"' \pc{399}{emptyf}, umgeben von einer Röhre aus
Glas, durch sie "`zugleich mit der Außenwelt verbunden und von ihr
abgeschnitten"' \pc{399}{emptyf}. Es ist Bettelheim, der den Namen
"`Connecticut-Indianer"', den Joey sich selbst gibt, in seine
Bestandteile "`connect"' und "`cut"' zerlegt \vgl{399}{emptyf} und
damit die Vorlage für \dg liefert.

\dg zergliedern das Wort im Anschluss daran allerdings nicht nur in
zwei sondern in drei Teile: \enquote{connect}, \enquote{I} und
\enquote{cut}. Und vielleicht lässt sich auch sagen, dass Bettelheim
den Ausdruck "`Connecticut"' eher der Seite der Katatonie, der
Abgetrenntheit von der Außenwelt, zuschlägt \vgl{403}{emptyf}, während
sich mit \dg die ganze Ambivalenz der Schizophrenie daran
verdeutlichen lässt.

In der Zerschneidung des Namens und dem Zusammenspiel seiner drei
Teile \emph{connect}, \emph{I} und \emph{cut} lässt sich
zeigen, was den schizophrenen Prozess ausmacht: Kontinuität im
Bruch, Kontinuität durch den Einschnitt hindurch, ja sogar das
\emph{I}, das Ich, das Subjekt, das "`ich"' sagt, in der Mitte,
gewissermaßen als Unterbrechung zwischen Verbindung (connect) und
Unterbrechung (cut). \emph{\cic} ist die "`Einheit von Spaltung
(schize) und Strom"' \pc{296}{ao}.
%, eine Einheit, die sich nur als Differenz denken lässt.

--

Besondere Bedeutung hat im Kontext der Schizophrenie diese Verbindung
von Verbindung und Unterbrechnung, von Strom und Einschnitt, von
\emph{connect} und \emph{cut}. Es sei hier erinnert an die
Formulierung, dass die Schizophrenie die "`Herstellung einer
nicht-lokalisierbaren Verbindung"' \pc{19}{schizg} sei. Verbindung ist
also nicht das, was einen bruchlosen Übergang herstellt, das, was
einfach weitergeht, und sorgt nicht für die Verschmelzung vorher
getrennter Teile. Umgekehrt zerschneidet der \emph{cut} nicht in
separate Teile, trennt nicht vollständig, nicht unwiederbringlich

Die besonderen Figuren, die in \emph{\cic} gefasst ist, sind gerade
die Verbindung von Verbindung und Unterbrechung, die Einheit von
Einheit und Differenz. Und das sind Figuren, die sich in der Analyse
im Folgenden auch als charakteristisch für das Kapital erweisen:
"`Kurz gesagt, der Begriff des Spaltungs-Stroms oder Strom-Einschnitts
schien uns den Kapitalismus wie die Schizophrenie gleichermaßen zu
bestimmen"' \pc{317}{ao}.

%"`die einzige identitätslose Einheit ist jene des Spaltungs-Stroms
%oder des Strom-Einschnitts"' \pc{314}{ao}.

%\worries{Einmal allgemein \cic und warum das schizophren
%ist.}

Macht man sich nun aber an diese Analyse und will man \gwg in einer
Weise reformulieren, die seine Schizophrenität nach außen treten
lässt, fällt zunächst auf, dass man es bei \gwg und \emph{\cic} nicht
mit einer einfachen Parallelität zu tun hat, in dem Sinne, dass sich
die Einzelteile der schizophrenen Prozesse eins zu eins einander
zuordnen ließen:

%\begin{center}
{\centering\hfill
\begin{tabular}{c@{ -- }c@{ -- }c}
Connect & I & cut\\
G & W & G'
\end{tabular}
\hfill}
%\end{center}

Das anfängliche Geld wäre hier die Verbindung, die größere Geldsumme
die Unterbrechung, und die Ware das Subjekt dieses Prozesses. Diese
Gegenüberstellung kann nicht der Schlüssel für das Verständnis der
schizophrenen Struktur auf beiden Seiten sein und würde die
Eigentümlichkeiten beider Formeln verfehlen. Stattdessen muss es darum
gehen, die ganze Bewegung des Kapitals im Sinne von \emph{\cic} zu
verstehen. Zunächst geht es deshalb um das Zusammenspiel von
Verbindung und Unterbrechung im Kreislauf des Kapitals, um
anschließend die Unterbrechung in Form der Ware genauer zu
untersuchen. Den Schluss macht die Frage nach dem Subjekt, dem
\emph{I} in \emph{\cic}.

% }}}

% Cut-Connect {{{

\subsubsection{Connect-Cut/Cut-Connect: Strom und Einschnitt}

%\worries{Das hier schon explizit für \gwg, dann auf die Ware kommen,
%die Ware sozusagen vorbereiten}

Nach der vorherigen Charakterisierung des Kapitals (s. Abschnitt \ref{gwg}) ist
die augenfälligste Parallelität sicherlich der Zusammenhang von
Verbindung und Unterbrechung im Kreislauf des Kapitals. \emph{Connect}
und \emph{cut}, die beiden Teile, die die Klammer des Ausdrucks \emph{\cic}
bilden, sind so gewissermaßen auch die Klammer des Kapitals.

-- Verbindung

Der Wert zirkuliert in einem Prozess, "`worin er Geldform und
Warenform bald annimmt, bald abstreift"' \pc{169}{kap}. Jeder Tausch,
der Teil von \gwg ist, also sowohl G--W als auch W--G ist einerseits
eine offensichtliche Verbindung. Jeder Tausch, jeder Formwechsel hält
den Gesamtkreislauf zusammen. Ohne sie gäbe es überhaupt nichts, das
zu verbinden wäre. Durch den Tausch werden die verschiedenen Formen
des Wertes verbunden und, wie ich später zeigen werde, die
Wertsteigerung, die doch zentral für das Kapital ist, überhaupt erst
ermöglicht.

-- Unterbrechung

Und andererseits ist jeder Übergang von Ware zu Geld und umgekehrt
eine Unterbrechung. Geld kann nicht einfach Geld und Ware nicht Ware
bleiben, um Kapital zu werden. Die Unterbrechung ihrer Existenz in der
jeweiligen Hand einer Besitzerin ist wiederum notwendige
Voraussetzung für die Zirkulation des Kapitals.

-- Beides zusammen

Aber über diese beiden Formwechsel hinweg, durch sie hindurch, steht
am Ende wieder Geld, die Form, "`wodurch seine Identität [die des
Werts] mit sich selbst konstatiert wird"' \pc{169}{kap}. Es ist
gewissermaßen eine unterbrochene Verbindung oder eine verbindende
Unterbrechung zwischen Ende und Anfang dieses Kreislaufs, an dem auch
schon Geld stand. Dieses Geld konnte aber nicht einfach Geld bleiben,
sondern musste ausgegeben, die gerade Linie zwischen Geld und Geld
musste unterbrochen werden, damit am Ende \emph{mehr} Geld stehen und
das Geld den Kreis somit \emph{als Kapital} durchlaufen konnte.

-- Übergang zur Ware

Nun ist die Bedingung der Differenz, die es dem Kapital überhaupt erst
möglich macht, zu sich zurückzukehren und nicht einfach unverändert
das gleiche Geld zu bleiben und nie Kapital zu werden, die
zwischenzeitliche Verausgabung des Geldes. Das Besondere ist also der
Umweg über die Ware. Nur durch diesen Umweg kommt das Kapital zu sich
zurück. Die Bedingungen dieser Möglichkeit müssen also im Mittelteil
dieses Dreischritts liegen, d.\,h. in der Ware.

--

%Das Kapital ... \worries{ZITATE}

%als Kontinuität im Bruch, als Identität im Von-sich-selbst-Abstoßen
%des Mehrwerts vom ursprünglichen Wert, Identität in der Differenz (zu
%sich selbst), also schizophrene Reise.

?

In der Überschreitung der eigenen inneren Grenzen und deren
unablässiger Reproduktion, dieser endlosen
Selbst-Ü\-ber\-schrei\-tung, "`erzeugt sich die Kontinuität des
kapitalistischen Prozesses in diesem stets verschobenen Einschnitt des
Einschnitts (coupure de coupure), anders gesagt in der Einheit von
Spaltung (schize) und Strom"' \pc{296}{ao}.

%... Ware vorbereiten

% }}}

% G-X-G' {{{
\subsubsection{G--X?--G': Die Bedeutung der Ware}

\worries{Nochmal Gegenüberstellung \gwg \cic. Ware nicht das Subjekt
usw.}

\worries{Ware doch Subjekt? Also zumindest das P, dass in der
Erweiterung in der Mitte auftaucht?}

-- Einstieg

"`In der ersten Form [W--G--W] vermittelt das Geld, in der anderen
[G--W--G'] umgekehrt die Ware den Gesamtverlauf"' \pc{163}{kap}, so
differenziert Marx die beiden Kreisläufe der Warenzirkulation und der
Kapitalzirkulation. Im Fall der allgemeinen Formel des Kapitals steht
die Ware in der Mitte (auch wenn die "`Vermittlung"', von der Marx
spricht, sicherlich nicht allein so buchstäblich zu verstehen ist), sie
unterbricht die gerade Linie zwischen Geld und mehr Geld. Sie
unterbricht sie nicht nur auf symbolischer Ebene, sondern ganz
handgreiflich, indem das Geld einmal ausgegeben, einmal gegen diese
Ware getauscht wird. Das Besondere dieser Ware muss untersucht werden,
will man verstehen, wie die Differenz zwischen G und G', also der
Mehrwert entsteht.

-- Nicht Übervorteilung

Marx betont mehrfach, dass der Mehrwert nicht einfach aus geschicktem
Tausch entstehen kann. Mehrwert-Produktion heißt nicht allseitige
Übervorteilung. Ihr Geheimnis liegt nicht im Gewinn, den die Händlerin
einstreicht, die eine Ware kauft und zu einem höheren Preis verkauft,
wie sich die Formel \gwg vielleicht auf den ersten Blick verstehen
ließe: "`Die Zirkulation oder der Warenaustausch schaffe keinen Wert"'
\pc{178}{kap}.
%\parencites[vgl.][170-181]{kap}[66]{kap2}

-- hinter dem Rücken

Es muss stattdessen etwas mit der Ware selbst, also im Schnitt, passieren,
etwas "`hinter [dem] Rücken"' der Zirkulation \pc{181}{kap}, wo die
Bewegung \gwg ansonsten stattfindet. Und um dem
Rechnung zu tragen, erweitert Marx die einfache Formel zu Beginn des
zweiten Bandes des \emph{Kapital}: Aus \gwg wird "`G--W \dots P \dots
W'--G', wo die Punkte andeuten, daß der Zirkulationsprozeß
unterbrochen ist"' \pc{31}{kap2} und "`P"' für produktives Kapital
steht \vgl{34}{kap2}.

Die Veränderung der Ware, mithin ihr Wertzuwachs, findet
\emph{außerhalb der Zirkulation} statt, dort, wo der
Zirkulationsprozess unterbrochen ist.

-- Unterbrechung

Marx hat hier außerdem die erste Bewegung der Zirkulation, G--W,
weiter zergliedert. Denn für die Produktion notwendig sind sowohl
Produktionsmittel (Pm) und die Arbeitskraft (A), die sie bedient
\vgl{32}{kap2}. Diese beiden Arten von Waren müssen also gekauft
werden, geschrieben als: \gwapm.

"`Aber das unmittelbare Resultat von \gwapm ist die Unterbrechung der
Zirkulation des in Geldform vorgeschoßnen Kapitalwerts"'
\pc{40}{kap2}.

Durch den Eintritt in die Produktion ist Zirkulation also unterbrochen
und trotzdem ist die Produktion notwendige Bedingung des
Kapitalkreislaufs: "`Die Bewegung stellt sich dar als \gwapmp, wo die Punkte andeuten,
daß die Zirkulation des Kapitals unterbrochen ist, sein
Kreislaufprozeß aber fortdauert, indem es aus der Sphäre der
Warenzirkulation in die Produktionssphäre eintritt"' \pc{40}{kap2}.

-- Arbeitskraft

Im Kern der Unterbrechung steht die Produktion. Und jetzt ist es
möglich, das Geheimnis der Ware, die in der Form \gwg die Rolle der
Vermittlung übernommen hat, zu lüften. Das Besondere ist die Ware
Arbeitskraft, deren Gebrauchswert Arbeit, also wertbildende Tätigkeit
ist: "`G--A ist das charakteristische Moment der Verwandlung von
Geldkapital in produktives Kapital, weil es die wesentliche Bedingung,
damit der in Geldform vorgeschoßne Wert sich wirklich in Kapital, in
Mehrwert produzierenden Wert verwandle"' \pc{35}{kap2}.

-- Ware schizo?

Die Anwendung dieser Ware Arbeitskraft \emph{außerhalb} der
Zirkulation ist der Schlüssel für die Wertsteigerung des Kapitals.
An dieser Stelle wird eine weitere Differenz eingezogen, die zwischen
Tauschwert und Gebrauchswert. Die Arbeitskraft ist eine Ware, "`deren
Gebrauchswert selbst die eigentümliche Beschaffenheit [besitzt],
Quelle von Wert zu sein"' \pc{181}{kap}.

Ohne Schizophrenie auf das Phänomen der gespaltenen Persönlichkeit
reduzieren zu wollen, ist doch die Ware selbst vielleicht schizophren.
Der Zusammenhang der zwei Seiten einer Ware, Tauschwert und
Gebrauchswert, lässt sich nicht nur als Spaltung begreifen, sondern
ebenso wiederum als Verbindung und Unterbrechung. Auf den ersten Blick
in zwei völlig verschiedene Sphären gehörig, etwa Zirkulation und
Konsumtion, bilden sie eine Einheit, namlich im Objekt selbst, das die
Ware ist, ohen dass doch die beiden Seiten dadurch weniger
unvereinbar würden.

Zwar sind beide notwendige Bestandteile dafür, dass eine Ware eine
Ware ist, aber es herrscht doch kein simpler kausaler Zusammenhang
zwischen ihnen, keine bruchlose Verbindung, Tauschwert und
Gebrauchswert lassen sich nicht einfach ineinander übersetzen. Weder
hängt, wie teuer eine Ware ist, direkt davon ab, wie nützlich sie ist,
noch umgekehrt. Aber eins lässt sich doch nicht ohne das andere haben,
solange die Ware eine Ware ist.

-- Außerhalb

Eine Bedingung der Identität des Kapitals liegt also immer außerhalb
ihrer selbst. Es ist der Produktionsprozess der Differenz, die in ihr
enthalten ist und die notwendig für diese Identität ist. Erst durch
die Rückverwandlung der Ware in Geld nach \glq getaner Arbeit\grq,
erst indem sich G' von G unterscheidet, wird die Identität offenbar.
Ohne Unterbrechnung, ohne Verwandlung in Ware, gäbe es auch keine
Identität, zumindest keine des Kapitals, denn das Geld bliebe einfach
Geld und würde sich nicht von der Stelle rühren. Von Kapital könnte
keine Rede sein.

% }}}

% Connect-I!-cut {{{

\subsubsection{Connect-I!-cut: Das Kapital als schizophrenes Subjekt}

Dieser Verweis nach außen bleibt für die Zirkulation immer
uneinholbar.

... scheint doch auch ein besonderes Abstammungsverhältnis involviert
zu sein: über eine Reihe von Zwischenschritten \glq produziert\grq{} G
G', aber G \grq produziert\grq{} nicht einfach irgendwelches anderes
Geld, sondern es produziert sich selbst als mehr Geld.

Zwar lassen sich verschiedene Teil einer Geldsumme unterscheiden ("`G'
= G + g"' \pc{51}{kap2}), so dass "`ein Teil einer Geldsumme als Mutter
eines anderen Teils derselben Geldsumme erscheint"' \pc{55}{kap2}. Am
Ende und als Voraussetzung des nächsten Kreislaufs ist diese
Unterscheidung aber wieder hinfällig: "`Aber als Resultat dieses
Kreislaufs G \dots G' existiert jetzt nur noch G'; es ist das Produkt,
worin sein Bildungsprozeß erloschen ist. G' existiert jetzt
selbständig für sich, unabhängig von der Bewegung, die es
hervorbrachte. Sie ist vergangen, es ist da an ihrer Stelle"'
\pc{49}{kap2}.

Eine Geldsumme, die in sich selbst eine Differenz erzeugt, die es
wiederum auslöschen muss. Diese Differenz von ursprünglichem Wert und
Mehrwert bildet die Grundlage und die Bedingung des neuen Werts, der
seinerseits sofort wieder nur \glq ursprünglicher\grq{} Wert eines
neuen Kreislaufs ist, sonst wäre er kein Kapital gewesen.

Spielerisch verpackt Marx das in christliche Terminologie. Der Sohn
erzeugt den Vater -- nur durch den Sohn ist der Vater Vater --, der
Mehrwert erzeugt das Kapital:
%
\begin{spacing}{1}
\begin{quote}
"`Er [der Wert] unterscheidet sich als ursprünglicher
Wert von sich selbst als Mehrwert, als Gott Vater von sich selbst als
Gott Sohn, und beide sind vom selben Alter und bilden in der Tat nur
eine Person, denn nur durch den Mehrwert von 10 Pfd. St. werden die
vorgeschossenen 100 Pfd. St. Kapital, und sobald sie dies geworden,
sobald der Sohn und durch den Sohn der Vater erzeugt, verschwindet ihr
Unterschied wieder und sind beide Eins, 110 Pfd. St."' \pc{S. 169
f.}{kap}.
\end{quote}
\end{spacing}

Das besondere Abstammungsverhältnis beinhaltet auch eine besondere
Zeitlichkeit. Ein bestimmtes "`Hinterher"' wird zur Bedingung der
Identität/Existenz \emph{in the first place}. Kapital ist nur jene
Wertsumme, die als eine größere Wertsumme zu sich selbst zurückkehrt,
oder, ausgedrückt in Geld, Geld, das als mehr Geld den Zirkel
schließt. Nur hinterher lässt sich sagen, ob das zu Beginn eingesetzte
Geld Kapital \emph{war}, als Kapital fungiert \emph{hat}. Das Kapital
"`ist immer sich selbst voraus"' \pc{125}{strauss}, um nicht zu sagen,
es hinkt sich immer selbst hinterher.

Als Erzeuger seiner selbst ist das Kapital Subjekt: "`der Wert [wird]
hier das Subjekt eines Prozesses, worin er unter dem beständigen
Wechsel der Formen von Geld und Ware seine Größe selbst verändert,
sich als Mehrwert von sich selbst als ursprünglichem Wert abstößt,
sich selbst verwertet"' \pc{169}{kap}.

% }}}

% Ausweg? {{{

\subsection{Schizophrenie als Ausweg?}

Um zum Ausgangspunkt zurückzukehren: Wie eingangs zitiert hatte Adorno
die Aufgabe gestellt, die "`innere Komposition"' \pc{261}{min} des
Subjekts aus den spezifischen Bedingungen der Produktionsweise
abzuleiten. Deleuze sprach von einer Diagnose "`in bezug auf äußerst
präzise Mechanismen ökonomischer, sozialer und politischer Natur"'
\pc{28}{schizg}.

Nach Analyse eines zentralen Mechanismus des Kapitalismus, der
zentralen Logik der Produktionsweise, nämlich der allgemeinen Formel
des Kapitals, lässt sich Deleuze und Guattaris Diagnose verstehen. Das
Kapital ist in seinem Kern schizophren. Deshalb die ist Schizophrenie
die Krankheit der heutigen Zeit. Und mit ihrer Diagnose sind \dg nicht
alleine. Auch Adorno erfüllt sich seine Aufgabe mit dem Urteil:
schizophren!

Auffällig ist jedoch, dass das Urteil der Schizophrenie für Adorno ein
durchweg negatives Urteil ist: "`Die im Individuum vollendete
Arbeitsteilung [\lips] kommt auf seine kranke Aufspaltung heraus"'
\pc{263}{min}. In der "`radikale[n] Objektivation"' \pc{263}{min}, als
Ergebnis dessen, dass "`der Prozeß, der mit der Verwandlung von
Arbeitskraft in Ware einsetzt, die Menschen samt und sonders
durchdringt und jede ihrer Regungen als eine Spielart des
Tauschverhältnisses a priori kommensurabel macht und
vergegenständlicht"' \pc{262}{min}, sieht Adorno "`die
gesellschaftliche Pathogenese der Schizophrenie"' \pc{263}{min}.

Hier ist kein fortschrittliches Potenzial in Sicht: Die
"`Selbsterhaltung verliert ihr Selbst"' \pc{263}{min}. Für die
"`Umorganisation"' \pc{263}{min} des Ich [für mehr was \worries{?}]
müssten die Menschen "`mit anwachsender Desintegration bezahlen"'
\pc{263}{min}. Statt einen Ausweg aufzuzeigen, deuten die ganzen
zersplitterten Subjekte für Adorno eher in Richtung Totalitarismus
\vgl{263}{min}.

--

Ganz anders fällt die Deutung der Diagnose bei \dg aus.
Schon oben wurde das Programm zitiert, die Schizophrenie in
irgendeiner Weise "`in ihrer Positivität und als Positivität"'
\pc{24}{schizg} zu begreifen. Um dem auf die Spur zu kommen, bedarf es
erneut der Deleuze'schen Überlegungen über das Verhältnis von
Kapitalismus und Schizophrenie.

Als "`generalisierte Decodierung der Ströme"', als "`neue
durchschlagende Deterritorialisierung"' \pc{288}{ao} ist der
Kapitalismus doch umgekehrt genauso auf stetige Recodierungen und
Reterritorialisierungen angewiesen: "`Letztlich ist es unmöglich,
Deterritorialisierung und Reterritorialisierung zu unterscheiden, da
sie sich wechselseitig enthalten oder die beiden Seiten ein und
desselben Prozesses ausmachen"' \pc{333}{ao}. Die eigenen inneren
Grenzen und Schranken, von Marx etwa im tendenziellen Fall der
Profitrate beschrieben, werden vom Kapitalismus beständig
überschritten, aber auch immer von neuem reproduziert.

Und eine äußere Grenze besitzt der Kapitalismus nicht, zumindest nicht aus
der Logik der Steigerung und Verwertung heraus. Aber hier bringen
Deleuze und Guattari wieder die Schizophrenie ins Spiel: "`Wir sagen
in einem, daß es eine äußere Grenze des Kapitalismus nicht gibt und
daß es sie doch gibt: nämlich die Schizophrenie, das heißt die
absolute Decodierung der Ströme, wenngleich der Kapitalismus allein
dadurch funktioniert, daß er diese Grenze zurückdrängt und abzuwenden
versucht"' \pc{322}{ao}.

Eine Bewegung der Decodierung und Deterritorialisierung, die sich auch
durch keine Axiomatik mehr einfangen lässt: so verstanden wird die
Schizophrenie zur Grenze des Kapitalismus. Und der Kapitalismus muss
weiter decodieren und weiter deterritorialisieren, sich weiter
steigern: "`Unaufhaltsam nähert sich seiner im eigentlichen Sinne
schizophrenen Grenze"' \pc{44}{ao}.

Hier wird deutlich, wie \dg darauf kommen, die Schizophrenie könnte
irgendein fortschrittliches Potenzial besitzen. Nicht Subversion,
nicht Reform, sondern schlicht unaufhaltsame Flucht, Sprengung aller
Grenzen. Das ist die Befreiungsperspektive der Schizophrenie, während
dem gegenüber Institutionen wie kapitalistische Staaten stehen, die
der völligen Schizophrenisierung aller Ströme entgegenarbeiten, um
"`zu reterritorialisieren, also zu verhindern, daß die decodierten
Ströme aus allen Öffnungen der gesellschaftlichen Axiomatik fliehen"'
\pc{332}{ao}.

Und dieses Fliehen ist der zentrale Punkt für das Verständnis des
Verhältnisses von Kapitalismus und Schizophrenie bei Deleuze und
Guattari. Der Kapitalismus deterritorialisiert, aber doch nicht völlig
bedingungslos und absolut, sondern bleibt auf eine gewisse Kontrolle
angewiesen. Er folgt noch der Logik der Verwertung. Die Schizophrenie
hingegen, ebenfalls Deterritorialisierung, hat das Potenzial zu
entfliehen, ohne immer neue Codes und Territorialitäten zu errichten.
Und so antworten Deleuze und Guattari auf die Frage nach der
Perspektive: weitermachen!
%
\begin{spacing}{1} \begin{quote} "`Aber welcher revolutionäre Weg, ist
überhaupt einer vorhanden? -- Sich [\lips] vom Weltmarkt zurückziehen,
in einer eigentümlichen Wiederaufnahme der faschistischen \glq
ökonomischen Lösung\grq? Oder den umgekehrten Weg einschlagen? Das
heißt mit noch mehr Verve sich in die Bewegung des Marktes, der
Decodierung und der Deterritorialisierung stürzen? denn vieleicht sind
die Ströme \emph{aus der Perspektive einer Theorie und Praxis der
zutiefst schizophrenen Ströme noch zuwenig decodiert und
deterritorialisiert}? Nicht vom Prozeß sich abwenden, sondern
unaufhaltsam weitergehen, \glq den Prozeß beschleunigen\grq, wie
Nietzsche sagte: wahrlich, in dieser Sache haben wir noch zuwenig
gesehen"' \pc{S. 308, meine Hervorh.}{ao} \worries{Ende doch eine
Relativierung? also raus?} \end{quote} \end{spacing}

\dg gehen bis zur apodiktisch anmutenden Gegenüberstellung von
"`Kapitalisten und Schizos"' \pc{328}{ao}.

Gerade mit Adorno im Hinterkopf muss an dieser Stelle aber die Frage
gestellt werden, ob das wirklich die Befreiungsperspektive sein kann.
Wie lässt sich das vorstellen? Ein Heer Schizophrener nimmt die
Bastionen des Kapitalismus im Sturm? Alle Schizos an die Waffen?

Soll die Perspektive wirklich in
vollständig kapitalistisch erschlossenen Subjekten liegen? In
Subjekten, die gewissermaßen sogar zu kapitalistisch, zu
deterritorialisiert für den Kapitalismus sind? Mit all den Problemen
der Selbst-Unterwerfung? Woran erkannt man, ob jemand gerade auf dem
Weg zur schizophrenen Grenze oder nur besonders kapitalistisch
gesteigert ist? Und was hieße es, für eine emanzipatorische Praxis,
wenn das das Gleiche ist? \dg stellen Kapitalisten nicht nur einander
gegenüber, sie attestieren ihnen nicht nur eine "`fundamental[e]
Feindschaft"' \pc{328}{ao}, sondern auch eine "`fundamental[e]
Verbundenheit"', nämlich "`auf der Ebene der Decodierung"'
\pc{328}{ao}. Wie lässt sich hier eine Grenze ziehen zwischen
der Möglichkeit der Transzendenz und der bloßen Affirmation
kapitalistischer Immanenz?

... die Frage, was das heißt, wenn man vorher die Schizophrenie bis in
den Kern des Kapitalismus zurückverfolgt hat.

%}}}

% Fazit {{{

\section{Fazit: Ausblicke und Fluchtlinien}

Innere Komposition aus der ökonomischen Struktur ableiten (Adorno),
Mechanismen (Deleuze) und so: Na, wenn die Grundstruktur schizophren
ist, dann müssen es ja auch die Subjekte sein.

"`unsere Gesellschaft produziert Schizos wie VWs ..."'

verschiedene schizophrene Subjekte

% }}}

\newpage
%\nocite{*}
\printshorthands
%\newpage
\printbibliography

\end{document}
