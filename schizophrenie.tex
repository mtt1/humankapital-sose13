% preamble {{{
\documentclass[12pt,
               DIV13,
               paper=a4,
               twoside=false,
               onehalfspacing,
               %titlepage,
               bibliography=totoc,
               toc=graduated,
               draft,
               ]{scrartcl}

\usepackage[utf8]{inputenc}
\usepackage[T1]{fontenc}
%\usepackage{ngerman}
%\usepackage[british]{babel}
\usepackage[ngerman]{babel}
\usepackage[babel,german=quotes]{csquotes}
\usepackage{setspace}
%\usepackage{mathptmx}           % pslatex's successor
%\usepackage[scaled=.92]{helvet} % pslatex's successor
%\usepackage{courier}            % pslatex's successor
\usepackage[osf]{libertine}
\usepackage{courier}
% GEHT NICHT?:
%\usefont{T1}{fxlj}{m}{n}\selectfont % Mit Zahlen, die nach unten hängen
\usepackage{color}
\usepackage{ifpdf}
\usepackage{scrpage2}
\usepackage{xspace}
\usepackage[babel=true]{microtype}

\usepackage[backend=biber,
            sortlocale=de,
            %style=authoryear,
            %citestyle=authoryear-ibid,
            citestyle=authoryear-icomp,
            bibstyle=authoryear,
            %natbib=true,
            sortlos=los,
            %autopunct=true,
            language=ngerman,
            clearlang=true,
            %babel=none,
            block=none,
            %ibidtracker=constrict, % automatically set by authoryear-icomp
            loccittracker=constrict, % no page-number after "ibidem" if the same page is cited again
            ]{biblatex}
\addbibresource{literatur.bib}

\ifpdf
 \usepackage[pdftex]{graphicx}
 \DeclareGraphicsExtensions{.pdf}
 \pdfcompresslevel=9
% \usepackage[%
%   pdftex=true,
%   backref=true,
%   linktocpage=true,
%   pdfpagemode=None
% ]{hyperref}
% \hypersetup{
%   pdftitle={},
%   pdfauthor={Matthias Rudolph},
%   pdfsubject={},
%   pdfcreator={LaTeX2e and pdfLaTeX},
%   pdfproducer={},
%   pdfkeywords={}
% }
\else
  \usepackage[dvips]{graphicx}
  \DeclareGraphicsExtensions{.eps}
\fi

%\setcounter{secnumdepth}{-1} % keine section-Nummerierung

\pagestyle{scrheadings}
%\automark[section]{subsection}

\ihead[]{}
\chead[]{TITEL} % Name der Hausarbeit
\ohead[]{Matthias Rudolph}
\ifoot[]{}
\cfoot[]{}
\ofoot[]{\thepage} % Seitenzahl

% Biblatex hacks
%
% Quick 'n' dirty
% Richtiger Artikel für "Hrsg. *vom* Institut"
\DefineBibliographyStrings{ngerman}{%
    bytranslator = {hrsg\adddotspace vom},
}

% Kein "Bd." in \volcite (trotzdem per Komma getrennt)
%\DeclareFieldFormat{volcitevolume}{#1}

% Doppelpunkt statt Punkt nach dem Label
\renewcommand{\labelnamepunct}{\addcolon\space}

% Schrägstriche zwischen mehreren AutorInnen
\renewcommand*{\multinamedelim}{\addslash}
\renewcommand*{\finalnamedelim}{\addslash}

% Alle Bibliografie-Einträge: Nachname, Vorname. Auch für Einträge
% mit mehreren AutorInnen und die Sigel-Liste.
% http://tex.stackexchange.com/questions/12806/guidelines-for-customizing-biblatex-styles
\DeclareNameAlias{sortname}{last-first} % Bibliografie
\DeclareNameAlias{default}{last-first} % Vollzitate (?)
%\DeclareNameAlias{labelname}{last-first} % alle anderen Zitate

% Nachname in Kapitälchen. Effekt aber nicht nur in der Bibliographie,
% sondern auch bei Zitaten.
%\renewcommand*{\mkbibnamelast}[1]{\textsc{#1}}

% More space between different entries in the bibliography.
% See \bibitemsep, \bibnamesep, \bibinitsep
% http://tex.stackexchange.com/questions/19105/how-can-i-put-more-space-between-bibliography-entries-biblatex
\setlength\bibnamesep{1em}

\newcommand{\freel}{\vspace{1em}}
\newcommand{\lips}{\dots\unkern}

\newcommand{\tit}[1]{\textit{#1}}
\newcommand{\tbf}[1]{\textbf{#1}}

\newcommand{\pc}[2]{\parencite[#1]{#2}}
\newcommand{\vgl}[2]{\parencite[vgl.][#1]{#2}}
\newcommand{\zn}[3]{\parencite[#1, zit. nach][#2]{#3}}

% Disable single lines at the start of a paragraph (Schusterjungen)
\clubpenalty = 10000

% Disable single lines at the end of a paragraph (Hurenkinder)
\widowpenalty = 10000
\displaywidowpenalty = 10000

% Don't insert more space at the end of a sentence than between words.
\frenchspacing

% Worries in Blau
\usepackage{ifdraft}
\newcommand{\worries}[1]{\ifdraft{\textcolor{blue}{\texttt{(#1)}}}{}}

% }}}

% Für diese Arbeit {{{

% Fürs Kapital
\newcommand{\gwg}{G--W--G'\xspace}
\newcommand{\wgw}{W--G--W\xspace}
\newcommand{\cic}{Connect-I-cut\xspace}

\newcommand{\hoe}{\tit{Homo oeconomicus}\xspace}

% Footnote tricks
% default:
%\deffootnote[1em]{1.5em}{1em}{%
%    \textsuperscript{\thefootnotemark}}

%\deffootnote[1em]{1em}{1em}{%
%    \textsuperscript{\thefootnotemark}}

%\deffootnote[1em]{1.5em}{1em}{\thefootnotemark\enspace}

%\deffootnote{1em}{1em}{\thefootnotemark\enspace}

% }}}

% Titel {{{

\begin{document}
\setcounter{page}{0}

\titlehead{Goethe-Universität Frankfurt am Main\\
Fachbereich Philosophie und Geschichtswissenschaften\\
Institut für Philosophie\\
Prof. Dr. Christoph Menke\\
Seminar: Demokratie und Kapitalismus,
SoSe 2013\\
Modul: VM 3b}
%\title{Das Subjekt des Humankapitals}
%\title{Die Selbstverwertung des Selbst/Subjekts}
\title{Xx}
%\subtitle{Zwischen Steigerungslogik und Schizophrenie}
\subtitle{Die schizophrene Struktur der Logik des Kapitals}
\author{Matthias Rudolph}
\date{Vorgelegt am: \today}

\maketitle
\vfill

\noindent Matthias Rudolph\\
Frankenallee 117\\
60326 Frankfurt/M\\
Matr.-Nr.: 5273120\\
Mod. Mag. Philosophie (7. FS), NF Soziologie (3. FS) \& Politikwissenschaft (2. FS)\\ % anpassen
mttrud@gmail.com
\newpage

\tableofcontents
\newpage

%}}}

% Einleitung {{{
\section{Einleitung}

Kombination zweier Thesen:

Krankheit unserer Zeit (Adorno und Deleuze)

Innere Komposition ableiten (Adorno)/Mechanismen (Deleuze)

% }}}

\section{Hauptteil}

\subsection{Schizophrenie und Kapitalismus}

% {{{ Schizo
\subsubsection{Schizophrenie}

Auch wenn die Rede von einem gespaltenen Subjekt den Aufhänger für
diesen Teil der Arbeit bildet, so ist doch die Diagnose der
Schizophrenie, die Deleuze und Guattari formulieren, nicht einfach in
der landläufigen Bedeutung als "`gespaltene Persönlichkeit"' zu
übersetzen. Auch bei Deleuze und Guattari steckt dahinter ein
medizinisch, psychologisch und philosophisch komplexes Konzept.

Schizophrenie gilt in der Fachliteratur als Krankheit ohne
einheitliches Krankheitsbild: "`\emph{Die} Schizophrenie als homogenes
Krankheitsbild mit einheitlichem klinischem Erscheinungsbild und einem
eindeutig vorhersagbaren Krankheitsverlauf mit immer wieder
vergleichbaren ähnlichen Krankheitsstadien gibt es nicht"'
\pc{799}{psych}.

Dennoch lässt sich eine gewisse Bandbreite schizophrenen Verhaltens
beschreiben: Die "`beiden Pole der Schizophrenie"' \pc{21}{schizg}
sind die katatonischen Krampfzustände und die wahnhafte Aktivität
oder, um es direkt in Deleuze'schen Begriffen zu sagen: "`Katatonie
des organlosen Körpers, anorganische Tätigkeit der Organmaschinen"'
\pc{21}{schizg}. Also auf der einen Seite eine "`Zersetzung der
Person"' und eine "`Abspaltung von der Realität, die einem starren und
sich selbst verschlossenen Innenleben eine Art Übergewicht oder
Autonomie verleihen"' \pc{23}{schizg} -- das, was üblicherweise
Autismus genannt wird \vgl{801}{psych}. Auf der anderen die
"`Zersplitterung oder funktionale Dislokation der Assoziationen"'
\pc{23}{schizg}, eine Betrachtung, "`die den fehlenden Zusammenhang
zur Hauptstörung macht"' \pc{23}{schizg}.

--

Anders als vielen gängigen Interpretationen geht es Deleuze und
Guattari aber darum, die Schizophrenie nicht im Sinne eines Mangels
(etwa von Kontinuität, Konsistenz, Sinn, Signifikant, Vater o.\,ä.
\vgl{xx}{schizg}) zu interpretieren, sie nicht auf die "`Merkmale des
Defizits"', der "` Zerstörung"', der "`Lücken und Spaltungen"'
\pc{24}{schizg} zu reduzieren. Stattdessen "`besteht die Schwierigkeit
darin, der Schizophrenie in ihrer Positivität und als Positivität
Rechnung zu tragen"' \pc{24}{schizg}.

Das heißt die Schizophrenie als Prozess verstehen, aber nicht bloßes
Fließen, etwa als kontinuierlichen Übergang zwischen verschiedenen
schizophrenen Stadien. Prozess selbst wird verstanden als "`Bruch, Einbruch,
Durchbruch, der die Kontinuität einer Persönlichkeit unterbricht und
sie auf eine Art Reise schickt, durch ein intensives und
erschreckendes \glq Mehr an Realität\grq{} hindurch, gemäß
Fluchtlinien, in denen Natur und Geschichte, Organismus und Geist sich
verfangen"' \pc{28}{schizg}. So ist die Schizophrenie die
"`Herstellung einer nicht-lokalisierbaren Verbindung"' \pc{19}{schizg}
zwischen heterogenen Elementen, so dass sie, "`\emph{gerade weil sie
keine Beziehung zueinander haben}, untereinander in Beziehung treten"'
\pc{19}{schizg}. Also Verbindung in der Unterbrechung, Strom durch den
Einschnitt hindurch, weshalb es Deleuze und Guattari tatsächlich auch
nicht einfach um "`gespaltene Persönlichkeiten"' geht: "`Spaltung ist
ein schlechtes Wort zur Bezeichnung des Zustands"' \pc{27}{schizg},
weil es nicht um zwei separate Einheiten geht, sondern um den "`Bruch,
Einbruch, Durchbruch"', um die Differenz in der Einheit selbst.

"`Das schizoide Werk par excellence"' \pc{54}{ao} sind "`Puzzleteile,
die aber nicht zu einem, sondern verschiedenen Puzzles gehören:
[\lips] mit ihren nicht zueinander passenden Rändern, die gewaltsam
ineinandergezwängt, ineinandergeschachtelt werden und stets Reste
übrig lassen"' \pc{54}{ao}.

"`Man kann sagen, daß der Schizophrene [\lips] \emph{alle Codes
durcheinanderbringt}"' \pc{22}{ao}. \worries{?}

"`Wenn die Schizophrenie als die Krankheit der heutigen Zeit
erscheint, dann nicht aufgrund von Allgemeinheiten, die unsere
Lebensweise betreffen, sondern in bezug auf äußerst präzise
Mechanismen ökonomischer, sozialer und politischer Natur"' \pc{28}{schizg}.

% }}}

% und Kap.mus {{{

\subsubsection{\dots und Kapitalismus}

"`In der Tat meinen wir, daß der Kapitalismus im Zuge seines
Produktionsprozesses eine ungeheure schizophrene Ladung erzeugt, auf
der wohl seine Repression lastet, die sich aber unaufhörlich als
Grenze des Prozesses reproduziert"' \pc{45}{ao}, so beschreiben
Deleuze und Guattari den Zusammenhang von Kapitalismus und
Schizophrenie. Die spezifischen Veränderungen, die sich mit dem
Aufkommen des Kapitals ergeben, bezeichnen sie dabei als Decodierung
und Deterritorialisierung -- zwei Begriffen, die im Kontext eines
besonderen Vokabulars der Maschinen usw. \worries{?} stehen.

Während frühere Gesellschaften (Deleuze und Guattari unterscheiden
territoriale, despotische und kapitalistische Gesellschaften bzw.
Gesellschaftsmaschinen \vgl{338}{ao}) gekennzeichnet waren von
klaren Territorialitäten (Ländern, Zünften, Ethnien usw.) und Codes
(z.\,B. der Zusammenhang von Konsumgütern und Prestige und die
Übersetzung des einen ins andere) \vgl{318, 332}{ao} bzw.
Übercodierung \worries{BSP}, sei das Besondere am Kapitalismus eben die allgemeine
Decodierung und Deterritorialisierung \vgl{337}{ao}: "`Denn jedes
Fließen des Stroms ist Deterritorialisierung, jede verschobene Grenze
Decodierung"' \pc{298}{ao}.

Was hier so vielleicht noch so undeutlich klingt, wird ganz
anschaulich zum Beispiel mit Blick auf Marx' Analyse der sogenannten
ursprünglichen Akkumulation: Kurz gesagt war es eine Bedingung für das
Entstehen des Kapitalismus, dass sich doppelt-freie Lohnarbeiter*innen
und Geld, das zu ihrer Bezahlung ausgegeben werden konnte, begegnet
sind \vgl{xx}{kap}. Deleuze fasst dies in Begriffen der Decodierung
und Deterritorialisierung, die für ihn den Kapitalismus kennzeichnen:
"`Tatsächlich entsteht er [der Kapitalismus] aus dem Zusammentreffen
zweier Arten von Strömen: den decodierten Produktionsströmen in Form
des Geld-Kapitals und den decodierten Arbeitsströmen in Form des \glq
freien Arbeiters\grq\,"' \pc{44}{ao}.

\worries{Aber immer wieder Axiomatik, neue Axiome}

\worries{Mehr: Organloser voller Körper usw.}

--

Zu den Mechanismen: "`Wir haben gesehen, daß das Verhältnis von
Kapitalismus und Schizophrenie bei weitem über die Probleme der
Lebensweise, der Umwelt, der Ideologie usw. hinausgeht, und auf der
grundlegenden Ebene ein und derselben Ökonomie, ein und desselben
Produktionsprozesses gesehen werden muß. Unsere Gesellschaft
produziert Schizos wie Haarwaschmittel oder wie VWs mit dem einzigen
Unterschied, daß jene nicht ver-|käuflich sind"' \pc{S. 315 f.}{ao}.

--> Lenger, das Erste ist das Dritte

---

Kapitalismus und Schizophrenie: beide Decodierung usw.

"`Ist es in diesem Sinne richtig zu sagen, daß die Schizophrenie das
Produkt der kapitalistischen Maschine sei, wie die depressive Manie
und die Paranoia Produkt der Despotenmaschine und die Hysterie Produkt
der Territorialmaschine?"' \pc{44}{ao}.

"`Im Differentialquotienten ist die grundlegende kapitalistische
Erscheinung zum Ausdruck gebracht: \emph{die Transformation des
Mehrwerts an Code in Mehrwert an Strömen}"' \pc{S. 292 f.}{ao}.

"`Die Tendenz besitzt einzig eine interne Grenze [aber keine äußere],
die sie überschreitet, allerdings indem sie sie verschiebt, das heißt
sie rekonstituiert, sie als interne Grenze, die erneut mittels
Verschiebung überschritten \worries{Selbst-Überschreitung} werden muß,
wiederfindet: so erzeugt sich die Kontinuität des kapitalistischen
Prozesses in diesem stets verschobenen Einschnitt des Einschnitts
(coupure de coupure), anders gesagt in der \emph{Einheit von Spaltung
(schize) und Strom}"' \pc{S. 296, meine Hervorh.}{ao}.

% }}}

% GWG {{{

\subsection{\gwg}

\gwg lautet die vielbeschworene Formel des Kapitals. Sie drückt aus,
wie auf dem Umweg über eine bestimmte Ware (W) aus Geld (G) mehr Geld
(G') wird. Nach der Marx'schen Arbeitswerttheorie ist das Kapital eine
Wertsumme, wobei allein Arbeit in der Lage ist, Wert zu schaffen
\vgl{181}{kap}. Allerdings ist das Kapital nicht einfach irgendeine
Wertsumme, es ist der Wert \emph{in Bewegung}, in eben der Bewegung,
die durch die Formel \gwg beschrieben wird.

%-- GWG vs WGW

%-- WGW

Wir befinden uns an dieser Stelle in der sogenannten
Zirkulationssphäre, in der Privateigentümer*innen Waren und Geld als
Äquivalente tauschen. Von der einen Seite stellt sich dieser Tausch
als Verwandlung von Ware in Geld (W--G), von der anderen Seite als
Verwandlung von Geld in Ware (G--W) dar. Diese Bausteine lassen sich
nun allerdings auf zwei verschiedene Weisen verketten, zum einen als
\wgw, zum anderen umgekehrt als G--W--G. Der erste Fall ließe sich als
"`Verkaufen, um zu kaufen"' beschreiben, der zweite als "`Kaufen, um
zu verkaufen"' \vgl{162}{kap}. Der entscheidene Unterschied liegt in
der damit ausgesprochenen Zielsetzung: Das Ziel von \wgw ist der
Austausch einer Ware gegen Geld, um damit eine andere Ware zu kaufen.
%"`Das Geld ist also definitiv ausgegeben"' \pc{163}{kap}.
Die Befriedigung eines Bedürfnisses mit der zweiten Ware, also ihr
Entzug aus der Zirkulation durch den Konsum, ist das Ziel. So besitzt
der Kreislauf \wgw einen ihm äußerlichen Zweck, und damit seine Grenze
und sein Ende. "`Konsumtion, Befriedigung von Bedürfnissen, mit einem
Wort, Gebrauchswert ist daher sein Endzweck"' \pc{164}{kap}.

%-- GWG

Anders mit G--W--G. Ziel ist hier nicht der Entzug einer Ware aus der
Zirkulation, und ebensowenig der Entzug des Geldes. Es wird vielmehr
ausgegeben, in die Zirkulation hineingeworfen, damit es am Ende
zurückkommt. Das Geld ist "`nur vorgeschossen"' \pc{163}{kap}. Nun
wäre die ganze Bewegung "`eine ebenso zwecklose als abgeschmackte
Operation"' \pc{165}{kap} und die Mühe sinnlos, stünde am Ende als
Resultat nur "`tautologisch"' \pc{164}{kap} das, womit begonnen wurde,
nachdem es obendrein einmal dem Risiko des Verlusts ausgesetzt war.
Der bloße Erhalt des Werts kann also nicht das Ziel sein. Ebensowenig
geht es aber um Erreichung eines vom Anfang qualitativ verschiedenen
Endpunkts. Denn: "`Eine Geldsumme kann sich von der andren Geldsumme
überhaupt nur durch ihre Größe unterscheiden. Der Prozeß G--W--G
schuldet seinen Inhalt daher keinem qualitativen Unterschied seiner
Extreme, denn sie sind beide Geld, sondern nur ihrer quantitativen
Verschiedenheit"' \pc{165}{kap}. Der Inhalt dieser Bewegung besteht
also darin, dass an ihrem Ende \emph{mehr} Geld steht als am Anfang.
An die Stelle des qualitativen Unterschieds der beiden Waren in \wgw
tritt hier ein quantitativer Unterschied der beiden
Geldsummen.\footnote{Das Rätsel, woher diese Differenz, d.\,h. diese
Vergrößerung der Wertsumme kommt, löst Marx durch die Einführung der
Differenz von Arbeit und Arbeitskraft und die Verschiebung des
Blickwinkels von der Zirkulation auf die Produktion. Die Ware
Arbeitskraft kann in der Zirkulation gekauft werden. Ihr spezifischer
Gebrauchswert ist aber Arbeit, also wertbildende Tätigkeit, die im
Produktionsprozess verausgabt wird \vgl{xx}{kap}. Die Betrachtung der
besonderen Ware Arbeitskraft, wie der ganzen Sphäre der Produktion,
lasse ich an dieser Stelle außen vor. Für die Bestimmung der
Steigerungslogik des Kapitals ist es unerheblich, woher die Steigerung
kommt, solange sie nur nicht als zufällige Nebenerscheinung verstanden
wird.\worries{?}} Aus der zweiten Art, die Bausteine G--W und W--G
zusammenzusetzen, wird also \gwg, mit G'$>$G.

%-- Steigerung / endlos / maßlos

Hier "`sind Anfang und Ende dasselbe, Geld, Tauschwert, und schon
dadurch ist die Bewegung endlos"' \pc{166}{kap}. Es handelt sich nicht
um einen einfachen Kreislauf, es kreist nicht immer unterschiedslos
die gleiche Wertsumme, sondern um eine Bewegung, die jeweils auf einer
höheren Stufe wiederholt wird. "`Das Ende jedes einzelnen Kreislaufs
[\lips] bildet daher von selbst den Anfang eines neuen Kreislaufs"'
\pc{S. 166 f.}{kap}. Und jedes Mal steht am Ende eine Vermehrung der
eingesetzten Wertsumme. Zum Ziel dieser endlos wiederholten Operation
wird einzig die Vermehrung des Geldes, aber eben nicht zum Entzug aus
der Zirkulation. Würde die Bewegung angehalten, "`hörten [die
beispielhaften 100 Pfd. St.] auf, Kapital zu sein. Der Zirkulation
entzogen, versteinern sie zum Schatz"' \pc{166}{kap}. "`Die
Zirkulation des Geldes als Kapital ist dagegen Selbstzweck, denn die
Verwertung des Werts existiert nur innerhalb dieser stets erneuerten
Bewegung"' \pc{167}{kap}.

%Dass am Ende eines Zyklus aus Geld mehr Geld geworden ist, bestimmt
%Marx als die spezifische Steigerungslogik des Kapitals. Dabei geht es
%nicht um die Einzelfälle, in denen geschickte Händlerinnen ungeschickte
%Käuferinnen übers Ohr hauen, sondern die Wertzunahme, die Produktion
%von Mehrwert, ohne die das Kapital kein Kapital ist.

%== Zu den Besonderheiten/Widersprüchlichkeiten dieser Bestimmung

%-- Form-Wechsel / Keine Substanz

Neben dieser spezifischen Steigerungslogik des Kapitals ist an dieser
Stelle aber insbesondere die Form der Bewegung von Interesse. Kapital
ist nicht einfach Geld, was aus sich selbst mehr Geld macht, wie im
"`Lapidarstil"' \pc{170}{kap} das zinstragende Kapital vorgestellt ist
und was zugegebenermaßen schon eine bemerkenswerte Eigenschaft wäre.
In seiner allgemeinen Form ist Kapital Geld, das über den Umweg seiner
Verausgabung als mehr Geld zu sich selbst zurückkommt. Es handelt sich
also nicht nur um eine Vermehrung, eine Steigerung, sondern obendrein
um eine Selbst-Verausgabung und eine Identität durch diese
Verausgabung hindurch. Das Kapital ist Kapital durch einen zweifachen
Wechsel der Form, einmal von Geld zu Ware und einmal von Ware zurück
zu Geld: "`beide, Ware und Geld, [funktionieren] nur als verschiedne
Existenzweisen des Werts selbst [\lips] Er geht beständig aus der
einen Form in die andre über, ohne sich in dieser Bewegung zu
verlieren"' \pc{S. 168 f.}{kap}.

%-- Kapital als Subjekt

Ein Satz von Helmut Reichelt müsste an dieser Stelle reformuliert
werden. Er sagte: "`am Ende der Darstellung wird sich zeigen, daß es
das Kapital selbst ist, das uns in verschiedenen Formen begegnet, die
sich alle als Momente seiner selbst erweisen"' \pc{181}{reichelt}.
Passender noch wäre es zu sagen, "`daß es das Kapital selbst ist, das
\emph{sich} in verschiedenen Formen begegnet"'. Selbst angetrieben,
die ständigen Verwandlungen überstehend, wird das Kapital zum
"`übergreifende[n] Subjekt"' \pc{169}{kap} seiner eigenen Verwertung.
Das Kapital ist das "`Subjekt der beschriebenen Bewegung, die es
selbst als sein eigner Verwertungsprozeß ist"'
\zn{Marx}{181}{reichelt}. In der Bewegung der Selbstverwertung nennt
Marx das Kapital schließlich "`automatisches Subjekt"' \pc{169}{kap},
es ist "`selbst der prozessierende Widerspruch"' \pc{601}{grundr}.

%"`, und verwandelt sich so in ein automatisches Subjekt"' \pc{S. 168 f.}{kap}.

%\worries{vs. S. 169, wo Marx den Wert "`Substanz"' nennt}

%Das Kapital "`ist immer sich selbst voraus, stets entgeht ein Rest der
%Zuschreibung"' \pc{125}{strauss}. Selbstüberschreitung

%"`In der Tat aber wird der Wert hier das Subjekt eines Prozesses,
%worin er unter dem beständigen Wechsel der Formen von Geld und Ware
%seine Größe selbst verändert, sich als Mehrwert von sich selbst als
%ursprünglichem Wert abstößt, sich selbst verwertet. Denn die Bewegung,
%worin er Mehrwert zusetzt, ist seine eigne Bewegung, seine Verwertung
%also Selbstverwertung"' \pc{169}{kap}.

%-- Widerspruch / Selbst-Negation

Am bemerkenswertesten ist aber eine Formulierung, die Helmut Reichelt
zitiert: "`Das Kapital ist daher in jeder besonderen Phase die
Negation seiner als des Subjekts der verschiednen Wandlungen"'
\zn{Marx}{181}{reichelt}. Diese eigentümliche Bewegung des Kapitals,
die über einen Umweg zu sich selbst zurückkommt, wird hier verstanden
als Negation.

Die Identität des Kapitals lässt sich nur noch bestimmen als
substanzlose Identität in der Differenz der Steigerung und der
zweifachen Verwandlung, ja der Selbst-Negation. \worries{?}

% das sich selbstnegierende Subjekt

% }}}

% Connect-I-cut {{{

\subsection{Die Schizophrenie der Logik: Connect-I-cut}

\worries{Einmal allgemein Connect-I-cut und warum das schizophren ist.
Dann: keine einfache Parallelität. Dann die drei Unterkapitel explizit
für \gwg}

Beinahe nebenbei werfen Deleuze und Guattari ein kleines Wortspiel
ein, das mir aber von besonderer Anschaulichkeit zu sein scheint.
Durch Einfügung von zwei Bindestriche, durch zwei Schnitte, wird aus
\emph{Connecticut} "`Connect -- I -- cut"' \pc{48}{ao} (wobei ich
dieses Wortspiel natürlich von seiner Territorialität, nämlich dem
US-Bundesstaat lösen möchte). In der Zerschneidung des
Namens und dem Zusammenspiel seiner drei Teile \emph{connect},
\emph{I} und \emph{cut} wird deutlich, was den schizophrenen Prozess
ausmacht: Kontinuität im Bruch, Kontinuität durch den Einschnitt
hindurch, ja sogar das \emph{I}, das Ich, das Subjekt, das "`ich"'
sagt, taucht an der Stelle zwischen Verbindung und Schnitt auf,
befindet sich gewissermaßen mitten im Schnitt. Connect -- I -- cut ist
die "`Einheit von Spaltung (schize) und Strom"' \pc{296}{ao}, eine
Einheit, die sich nur als Differenz denken lässt.

Und genau in diesem Sinne lässt sich jetzt auch \gwg verstehen: als
Kontinuität im Bruch, als Identität im Von-sich-selbst-Abstoßen des
Mehrwerts vom ursprünglichen Wert, Identität in der Differenz (zu sich
selbst), also schizophrene Reise. In der Überschreitung der eigenen
inneren Grenzen und deren unablässiger Reproduktion, dieser endlosen
Selbst-Ü\-ber\-schrei\-tung, "`erzeugt sich die Kontinuität des
kapitalistischen Prozesses in diesem stets verschobenen Einschnitt des
Einschnitts (coupure de coupure), anders gesagt in der Einheit von
Spaltung (schize) und Strom"' \pc{296}{ao}.

--

Keine einfache Parallelität, im Sinne von:

\begin{tabular}{c@{ - }c@{ - }c}
Connect & I & cut\\
G & W & G'
\end{tabular}

Diese Gegenüberstellung würde die Eigentümlichkeiten beider Formeln
verfehlen. Stattdessen muss es darum gehen, die ganze Bewegung des
Kapitals \gwg im Sinne von \cic zu verstehen.

% }}}

% Cut-Connect {{{

\subsubsection{Connect-Cut/Cut-Connect: Strom und Einschnitt}

\worries{Das hier schon explizit für \gwg, dann auf die Ware kommen,
die Ware sozusagen vorbereiten}

Besondere Bedeutung hat im Kontext der Schizophrenie die Verbindung
von Verbindung und Unterbrechnung, von Strom und Einschnitt, von
Connect und Cut.

"`die einzige identitätslose Einheit ist jene des Spaltungs-Stroms
oder des Strom-Einschnitts"' \pc{314}{ao}.

"`Kurz gesagt, der Begriff des Spaltungs-Stroms oder Strom-Einschnitts
schien uns den Kapitalismus wie die Schizophrenie gleichermaßen zu
bestimmen"' \pc{317}{ao}.

% }}}

% G-X-G' {{{
\subsubsection{G--X?--G': Die Bedeutung der Ware}

"`In der ersten Form [\wgw] vermittelt das Geld, in der anderen [\gwg]
umgekehrt die Ware den Gesamtverlauf"' \pc{163}{kap}.

Ware als Einschnitt.

Im Schnitt passiert etwas "`hinter ihrem Rücken"' \pc{181}{kap} (dem
Rücken der Zirkulation, wo die Bewegung \gwg ansonsten spielt).

Außerhalb der Zirkulation, als im Schnitt, setzt die Ware Wert zu.
Hier findet die Produktion von Mehrwert statt, ohne dass in der
Zirkulation jemand übervorteilt würde. Der Schlüssel dafür ist die
besondere Ware Arbeitskraft, deren Gebrauch wertbildende Tätigkeit,
also Arbeit ist.

Die Existenzbedingungen des Kapitals liegen also außerhalb des
eigentlichen Kapital-Kreislaufs. Erst mit der Rückverwandlung der Ware
in G' sind sie erfüllt und das erste G \emph{war} Kapital. Dafür
braucht es den Schnitt im Strom.

Die Bedinungen liegen also außerhalb und erfüllen sich nur
nachträglich. Ohnehin handelt es sich um die Identitätsbedingungen
einer Bewegung, nicht von einer Substanz. Es geht also um die
Identität einer Bewegung, einer unterbrochenen Bewegung, einer
Identität im Schnitt und durch den Schnitt, also wegen des Schnitts
und durch ihn hindurch.

Ohne Unterbrechnung, ohne Verwandlung in Ware, gäbe es auch keine
Identität, zumindest keine des Kapitals, denn das Geld bliebe einfach
Geld und würde sich nicht von der Stelle rühren. Von Kapital könnte
keine Rede sein.

% }}}

% Connect-I!-cut {{{

\subsubsection{Connect-I!-cut: Das Kapital als schizophrenes Subjekt}

Die Bewegung des Kapitals, also \gwg im Gegensatz zu \wgw verliere
"`Sinn und Verstand"' \pc{166}{kap}, sagte schon Marx. Dabei geht es
um den Wert als Subjekt seiner eigenen Verwertung: "`In der Tat aber
wird der Wert hier das Subjekt eines Prozesses, worin er unter dem
beständigen Wechsel der Formen von Geld und Ware seine Größe selbst
verändert, sich als Mehrwert von sich selbst als ursprünglichem Wert
abstößt, sich selbst verwertet"' \pc{169}{kap}. Der Wert soll sich
also als Mehrwert von sich selbst als ursprünglichem Wert abstoßen.
Von einer einfachen Identität kann hier jedenfalls keine Rede sein.
Vielmehr gibt es bereits drei Einheiten, Wert, ursprünglicher Wert und
Mehrwert, die irgendwie zueinander in Beziehung stehen. Diese
Beziehung ist eine Beziehung der Zusammenfassung im Wert, dem neuen
Wert, als auch eine der Abstoßung, des neuen Werts als Mehrwert über
den alten.

Außerdem spielt eine besondere Zeitlichkeit eine Rolle. Ein bestimmtes
Hinterher wird zur Bedingung der Identität \emph{in the first place}.
Deutlicher wird dies, wenn die Bewegung noch einmal in Begriffen des
Kapitals reformuliert wird. Kapital ist nur jene Wertsumme, die als
eine größere Wertsumme zu sich selbst zurückkommt, oder, ausgedrückt
in Geld, Geld, das als mehr Geld den Zirkel schließt. \worries{Falte}
Hier wird deutlich, dass die Identitätsbedingungen des Kapitals sich
immer nur hinterher realisiert haben. Nur hinterher lässt sich sagen,
ob das zu Beginn eingesetzte Geld Kapital \emph{war}, als Kapital
fungiert \emph{hat}. Das Kapital "`ist immer sich selbst voraus"'
\pc{125}{strauss}. Und diese Differenz von ursprünglichem Wert und
Mehrwert bildet die Grundlage und die Bedingung des neuen Werts, der
seinerseits sofort wieder nur "`ursprünglicher"' Wert eines neuen
Kreislaufs ist, sonst wäre er kein Kapital gewesen:
%
\begin{spacing}{1}
\begin{quote}
"`Er [der Wert] unterscheidet sich als ursprünglicher
Wert von sich selbst als Mehrwert, als Gott Vater von sich selbst als
Gott Sohn, und beide sind vom selben Alter und bilden in der Tat nur
eine Person, denn nur durch den Mehrwert von 10 Pfd. St. werden die
vorgeschossenen 100 Pfd. St. Kapital, und sobald sie dies geworden,
sobald der Sohn und durch den Sohn der Vater erzeugt, verschwindet ihr
Unterschied wieder und sind beide Eins, 110 Pfd. St."' \pc{S. 169
f.}{kap}.
\end{quote}
\end{spacing}

% }}}

\subsection{xxx}

% Fazit {{{

\section{Fazit: Ausblicke und Fluchtlinien}

verschiedene schizophrene Subjekte

% }}}

\newpage
%\nocite{*}
\printshorthands
%\newpage
\printbibliography

\end{document}
