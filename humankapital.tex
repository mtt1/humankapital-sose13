% preamble {{{
\documentclass[12pt,
               DIV13,
               paper=a4,
               twoside=false,
               onehalfspacing,
               %titlepage,
               bibliography=totoc,
               toc=graduated,
               draft,
               ]{scrartcl}

\usepackage[utf8]{inputenc}
\usepackage[T1]{fontenc}
%\usepackage{ngerman}
%\usepackage[british]{babel}
\usepackage[ngerman]{babel}
\usepackage[babel,german=quotes]{csquotes}
\usepackage{setspace}
%\usepackage{mathptmx}           % pslatex's successor
%\usepackage[scaled=.92]{helvet} % pslatex's successor
%\usepackage{courier}            % pslatex's successor
\usepackage[osf]{libertine}
\usepackage{courier}
% GEHT NICHT?:
%\usefont{T1}{fxlj}{m}{n}\selectfont % Mit Zahlen, die nach unten hängen
\usepackage{color}
\usepackage{ifpdf}
\usepackage{scrpage2}
\usepackage{xspace}

\usepackage[backend=biber,
            sortlocale=de,
            %style=authoryear,
            %citestyle=authoryear-ibid,
            citestyle=authoryear-icomp,
            bibstyle=authoryear,
            %natbib=true,
            sortlos=los,
            %autopunct=true,
            language=ngerman,
            clearlang=true,
            %babel=none,
            block=none,
            %ibidtracker=constrict, % automatically set by authoryear-icomp
            loccittracker=constrict, % no page-number after "ibidem" if the same page is cited again
            ]{biblatex}
\addbibresource{literatur.bib}

\ifpdf
 \usepackage[pdftex]{graphicx}
 \DeclareGraphicsExtensions{.pdf}
 \pdfcompresslevel=9
% \usepackage[%
%   pdftex=true,
%   backref=true,
%   linktocpage=true,
%   pdfpagemode=None
% ]{hyperref}
% \hypersetup{
%   pdftitle={},
%   pdfauthor={Matthias Rudolph},
%   pdfsubject={},
%   pdfcreator={LaTeX2e and pdfLaTeX},
%   pdfproducer={},
%   pdfkeywords={}
% }
\else
  \usepackage[dvips]{graphicx}
  \DeclareGraphicsExtensions{.eps}
\fi

%\setcounter{secnumdepth}{-1} % keine section-Nummerierung

\pagestyle{scrheadings}
%\automark[section]{subsection}

\ihead[]{}
\chead[]{TITEL} % Name der Hausarbeit
\ohead[]{Matthias Rudolph}
\ifoot[]{}
\cfoot[]{}
\ofoot[]{\thepage} % Seitenzahl

% Biblatex hacks
%
% Quick 'n' dirty
% Richtiger Artikel für "Hrsg. *vom* Institut"
\DefineBibliographyStrings{ngerman}{%
    bytranslator = {hrsg\adddotspace vom},
}

% Kein "Bd." in \volcite (trotzdem per Komma getrennt)
%\DeclareFieldFormat{volcitevolume}{#1}

% Doppelpunkt statt Punkt nach dem Label
\renewcommand{\labelnamepunct}{\addcolon\space}

% Schrägstriche zwischen mehreren AutorInnen
\renewcommand*{\multinamedelim}{\addslash}
\renewcommand*{\finalnamedelim}{\addslash}

% Alle Bibliografie-Einträge: Nachname, Vorname. Auch für Einträge
% mit mehreren AutorInnen und die Sigel-Liste.
% http://tex.stackexchange.com/questions/12806/guidelines-for-customizing-biblatex-styles
\DeclareNameAlias{sortname}{last-first} % Bibliografie
\DeclareNameAlias{default}{last-first} % Vollzitate (?)
%\DeclareNameAlias{labelname}{last-first} % alle anderen Zitate

% Nachname in Kapitälchen. Effekt aber nicht nur in der Bibliographie,
% sondern auch bei Zitaten.
%\renewcommand*{\mkbibnamelast}[1]{\textsc{#1}}

% More space between different entries in the bibliography.
% See \bibitemsep, \bibnamesep, \bibinitsep
% http://tex.stackexchange.com/questions/19105/how-can-i-put-more-space-between-bibliography-entries-biblatex
\setlength\bibnamesep{1em}

\newcommand{\freel}{\vspace{1em}}
\newcommand{\lips}{\dots\unkern}

\newcommand{\tit}[1]{\textit{#1}}
\newcommand{\tbf}[1]{\textbf{#1}}

\newcommand{\pc}[2]{\parencite[#1]{#2}}
\newcommand{\vgl}[2]{\parencite[vgl.][#1]{#2}}
\newcommand{\zn}[3]{\parencite[#1, zit. nach][#2]{#3}}

% Disable single lines at the start of a paragraph (Schusterjungen)
\clubpenalty = 10000

% Disable single lines at the end of a paragraph (Hurenkinder)
\widowpenalty = 10000
\displaywidowpenalty = 10000

% Don't insert more space at the end of a sentence than between words.
\frenchspacing

% Worries in Blau
\usepackage{ifdraft}
\newcommand{\worries}[1]{\ifdraft{\textcolor{blue}{\texttt{(#1)}}}{}}

% }}}

% Für diese Arbeit {{{

% Fürs Kapital
\newcommand{\gwg}{G--W--G'\xspace}
\newcommand{\wgw}{W--G--W\xspace}

\newcommand{\hoe}{\tit{Homo oeconomicus}\xspace}

% Footnote tricks
% default:
%\deffootnote[1em]{1.5em}{1em}{%
%    \textsuperscript{\thefootnotemark}}

%\deffootnote[1em]{1em}{1em}{%
%    \textsuperscript{\thefootnotemark}}

%\deffootnote[1em]{1.5em}{1em}{\thefootnotemark\enspace}

%\deffootnote{1em}{1em}{\thefootnotemark\enspace}

% }}}

% Titel {{{

\begin{document}
\setcounter{page}{0}

\titlehead{Goethe-Universität Frankfurt am Main\\
Fachbereich Philosophie und Geschichtswissenschaften\\
Institut für Philosophie\\
Prof. Dr. Christoph Menke\\
Seminar: Demokratie und Kapitalismus,
SoSe 2013\\
Modul: VM 3b}
%\title{Das Subjekt des Humankapitals}
\title{Die Selbstverwertung des Selbst/Subjekts}
%\subtitle{Zwischen Steigerungslogik und Schizophrenie}
\subtitle{Humankapital und Schizophrenie}
\author{Matthias Rudolph}
\date{Vorgelegt am: \today}

\maketitle
\vfill

\noindent Matthias Rudolph\\
Frankenallee 117\\
60326 Frankfurt/M\\
Matr.-Nr.: 5273120\\
Mod. Mag. Philosophie (7. FS), NF Soziologie (3. FS) \& Politikwissenschaft (2. FS)\\ % anpassen
mttrud@gmail.com
\newpage

\tableofcontents
\newpage

%}}}

\section{Einleitung}

\section{Hauptteil}

\subsection{Humankapital}

[EINLEITUNG DES HAUPTTEILS]

In seinen Gouvernementalitätsstudien \parencites[vgl.][]{stb}{gbp},
aber auch schon im fünften Kapital von \citetitle{wzw} \pc{}{wzw}
analysiert und diagnostiziert Michel Foucault das Aufkommen eines
neuen Macht-Typs, der Bio-Macht, nach, aber ebenso  in, über und
neben, der Disziplinar-Macht. \worries{Beide auf das Leben gerichtet?
WzW, 134; und die sich folgendermaßen unterscheiden ...} Ebenso
beschäftigt ihn dabei die Frage, wie ein Subjekt beschaffen sein muss,
damit es sich auf diese Art und Weise regieren lässt. In
\tit{Vorlesung 11} von \citetitle{gbp} \vgl{367-398}{gbp} analysiert
Foucault deshalb die Figur des \hoe als eine spezifische Subjektform,
die er bis zum englischen Empirismus zurückverfolgt, deren
Wiederaufleben in ökonomischen Theorien neoliberaler Prägung ihn aber
besonders interessiert. Er findet den "`\hoe als Partner, als
Gegenüber, als Basiselement der neuen gouvernementalen Vernunft, wie
sie sich im 18. Jahrhundert ausbildet"' \pc{372}{gbp}.

In diesem Kontext betrachtet Foucault die Ausweitung ökonomischer
Analysen auf Bereiche, die zunächst nicht ökonomisch scheinen oder
zumindest bisher nicht als ökonomisch betrachtet wurden, etwa Analysen
der Kriminalität oder der Ehe \vgl{367}{gbp}. Ausgeweitet auf 
die Arbeit führen diese neoliberalen Analysen auf eine Vorstellung der
Menschen als Humankapital.

Sie beginnen mit der Frage, warum die Menschen arbeiten. Die Antwort,
die gefunden wird, lautet: weil sie dafür ein Einkommen, nämlich Lohn,
erhalten. Was aber ist ein Einkommen? Und die neoliberale Antwort ist:
"`Ein Einkommen ist ganz einfach das Ergebnis oder der Ertrag eines
Kapitals"' \pc{S. 311 f.}{gbp}. Aber aus welchem Kapital soll das
Einkommen einer Arbeiterin entspringen -- insbesondere mit Blick auf die
klassische Marx'sche Unterscheidung zwischen doppel-freien
Lohnarbeiterinnen und Kapitalbesitzerinnen? Foucault paraphrasiert die
Antwort als "`Gesamtheit aller physischen, psychologischen usw.
Faktoren, die jemanden in die Lage versetzen, einen bestimmten Lohn zu
verdienen [\lips] d.\,h. eine Fähigkeit, eine Kompetenz"' \pc{312}{gbp}.

Auffällig ist die damit verbundene Weite der Definition: "`\,\glq
Kapital\grq{} [ist] alles, was auf die eine oder andere Weise eine
Quelle von zukünftigem Einkommen sein kann"' \pc{312}{gbp}. Nicht nur
unterläuft die Analyse so die gängige Differenz und Frontstellung
zwischen Kapital und Arbeit, die Arbeit selbst wird verdoppelt in
Kapital und Einkommen \vgl{312}{gbp}.

Andererseits lässt sich das Kapital einer Arbeiterin, das ihre
Arbeitsfähigkeit ist, gleichwohl nicht auf dieselbe Art verstehen wie
die beiden anderen Quellen von Einkommen, die üblicherweise angeführt
werden und die Marx alle drei im Sprichwort von der trinitarischen
Formel \vgl{Kapitel 48, S. 822-839}{kap3} \worries{?} zusammengefasst
hat: Im Unterschied zu Maschinen und Boden hängt die Arbeitsfähigkeit
an der Person, deren Fähigkeit sie ist: sie kann "`nicht von der
menschlichen Person als ihrem Träger getrennt werden"' \pc{315}{gbp}.
Dieses Kapital Arbeit ist also nicht einfach Kapital, es ist
\emph{Human}kapital.

Mit daran anknüpfenden Konzepten, etwa von Investition ins (eigene)
Humankapital, die dann auf eine weite Palette menschlichen Verhaltens
angewandt werden kann -- von der Zeit, die Eltern mit ihren Kindern
verbringen, über Gesundheitsvorsorge bis zur Mobilität, etwa der
Bereitschaft umzuziehen \vgl{320}{gbp} --, versucht die neoliberale
Theorie, menschliches Verhalten, das allgemein ökonomisch als
Verteilung "`knappe[r] Ressourcen auf alternative Zwecke"'
\pc{310}{gbp} verstanden wird, zu erklären.

Diese Zusammenlegung von Kapital und Mensch, oder noch spezifischer
diese In-eins-Set\-zung von Kapital und Arbeit, scheint aber bereits
auf den ersten Blick problematische theoretische \worries{und
praktische?} Folgen für das vorgestellte Subjekt zu haben. Im
folgenden werde ich deshalb die Konsequenzen untersuchen, die
Bewegungen der Selbsterschließung, der Selbstverwertung, der
Selbststeigerung, die von der Vorstellung des Humankapitals impliziert
sind, für das Subjekt haben. Zunächst geht es aber um die ökonomische
Fundierung des Begriffs des Humankapitals. Dafür Karl Marx ...

\worries{Warum Marx?}

\subsection{Kapital}

\subsubsection{Kapital bei Marx}

\gwg lautet die vielbeschworene Formel des Kapitals. Sie drückt aus, wie
auf dem Umweg über eine bestimmte Ware (W) aus Geld (G) mehr Geld (G')
wird. Nach der Marx'schen Arbeitswerttheorie ist das Kapital eine
Wertsumme, wobei allein Arbeit in der Lage ist, Wert zu schaffen.
Allerdings ist das Kapital nicht einfach irgendeine Wertsumme, es ist
der Wert \emph{in Bewegung}, in eben der Bewegung, die durch die
Formel \gwg beschrieben wird.

-- GWG vs WGW

-- WGW

Wir befinden uns an dieser Stelle in der sogenannten
Zirkulationssphäre, in der Ware und Geld von Privateigentümer*innen
als Äquivalente getauscht werden. Aus der einen Sicht stellt sich
dieser Tausch als Verwandlung von Geld in Ware (G--W), aus der anderen
Sicht als Verwandlung von Ware in Geld (W--G) dar. Diese Bausteine
lassen sich nun allerdings auf zwei verschiedene Weisen verketten, zum
einen in der Bewegung \wgw, zum anderen umgekehrt als G--W--G. Der
erste Fall ließe sich als "`Verkaufen, um zu kaufen"' beschreiben, der
zweite als "`Kaufen, um zu verkaufen"'. Der entscheidene Unterschied
liegt in der damit ausgesprochenen Zielsetzung: Das Ziel von \wgw ist
der Austausch einer Ware gegen Geld, um damit eine andere Ware zu
kaufen.
%"`Das Geld ist also definitiv ausgegeben"' \pc{163}{kap}.
Die Befriedigung eines Bedürfnisses mit der zweiten Ware, also ihr
Entzug aus der Zirkulation durch den Konsum, ist das Ziel. So erhält
der Kreislauf \wgw einen ihm äußerlichen Zweck, und damit seine Grenze
und sein Ende. "`Konsumtion, Befriedigung von Bedürfnissen, mit einem
Wort, Gebrauchswert ist daher sein Endzweck"' \pc{164}{kap}.

-- GWG

Anders bei G--W--G. Ziel ist hier nicht der Entzug einer Ware aus der
Zirkulation, und ebensowenig der Entzug des Geldes. Es wird vielmehr
ausgegeben, in die Zirkulation hineingeworfen, damit es am Ende
zurückkommt. Das Geld ist "`nur vorgeschossen"' \pc{163}{kap}. Nun
wäre die ganze Bewegung "`eine ebenso zwecklose als abgeschmackte
Operation"' \pc{165}{kap} und die Mühe sinnlos, stünde am Ende nur das
"`tautologisch[e]"' \pc{164}{kap} Resultat, mit dem begonnen wurde,
nachdem es obendrein einmal dem Risiko des Verlusts ausgesetzt und
gegen eine Ware ausgetauscht wurde. Der bloße Erhalt des Werts kann
also nicht das Ziel sein. Ebensowenig geht es aber um Erreichung eines
vom Anfang qualitativ verschiedenen Endpunkts. Denn: "`Eine Geldsumme
kann sich von der andren Geldsumme überhaupt nur durch ihre Größe
unterscheiden. Der Prozeß G--W--G schuldet seinen Inhalt daher keinem
qualitativen Unterschied seiner Extreme, denn sie sind beide Geld,
sondern nur ihrer quantitativen Verschiedenheit"' \pc{165}{kap}. Der
Inhalt dieser Bewegung besteht also darin, dass an ihrem Ende
\emph{mehr} Geld steht als am Anfang. An die Stelle des qualitativen
Unterschieds der beiden Waren in \wgw tritt hier ein quantitativer
Unterschied der beiden Geldsummen.\footnote{Das Rätsel, woher diese
Differenz, d.\,h. diese Vergrößerung der Wertsumme kommt, löst Marx
durch die Einführung der Differenz von Arbeit und Arbeitskraft und die
Verschiebund des Blickwinkels von der Zirkulation auf die Produktion.
Die Ware Arbeitskraft kann in der Zirkulation gekauft werden. Ihr
spezifischer Gebrauchswert ist aber Arbeit, also wertbildende
Tätigkeit, die im Produktionsprozess verausgabt wird \vgl{xx}{kap}.
Die Betrachtung der besonderen Ware Arbeitskraft, wie der ganzen
Sphäre der Produktion, lasse ich an dieser Stelle außen vor. Für die
Bestimmung der Steigerungslogik des Kapitals ist es unerheblich, woher
die Steigerung kommt, solange sie nur nicht als zufällige
Nebenerscheinung verstanden wird.}

-- Steigerung / endlos / maßlos

Neben \wgw wird aus der zweiten Art die Bausteine G--W und W--G
zusammenzusetzen also \gwg, mit G'$>$G. Hier "`sind Anfang und Ende
dasselbe, Geld, Tauschwert, und schon dadurch ist die Bewegung
endlos"' \pc{166}{kap}. Es handelt sich nicht um
einen einfachen Kreislauf, es kreist nicht immer unterschiedslos die
gleiche Wertsumme, sondern um eine Bewegung, die jeweils auf
einer höheren Stufe wiederholt wird. "`Das Ende jedes einzelnen Kreislaufs [\lips]
bildet daher von selbst den Anfang eines neuen Kreislaufs"' \pc{S. 166
f.}{kap}. Und jedes Mal steht am Ende eine Vermehrung der
eingesetzten Wertsumme. Zum Ziel dieser endlos wiederholten Operation
wird einzig die Vermehrung des Geldes, aber eben nicht zum Entzug aus
der Zirkulation. Würde die Bewegung angehalten, "`hörten [die
beispielhaften 100 Pfd. St.] auf, Kapital zu sein. Der Zirkulation
entzogen, versteinern sie zum Schatz"' \pc{166}{kap}.

Ohne Ende ist "`die Zirkulation des Geldes als Kapital [\lips] dagegen
Selbstzweck, denn die Verwertung des Werts existiert nur innerhalb
dieser stets erneuerten Bewegung. Die Bewegung des Kapitals ist daher
maßlos"' \pc{167}{kap}.

%Dass am Ende eines Zyklus aus Geld mehr Geld geworden ist, bestimmt
%Marx als die spezifische Steigerungslogik des Kapitals. Dabei geht es
%nicht um die Einzelfälle, in denen geschickte Händlerinnen ungeschickte
%Käuferinnen übers Ohr hauen, sondern die Wertzunahme, die Produktion
%von Mehrwert, ohne die das Kapital kein Kapital ist.

== Zu den Besonderheiten/Widersprüchlichkeiten dieser Bestimmung

-- Form-Wechsel / Keine Substanz

Neben dieser spezifischen Steigerungslogik des Kapitals ist an dieser
Stelle aber insbesondere die Form der Bewegung von Interesse. Kapital
ist nicht einfach Geld, was aus sich selbst mehr Geld macht, wie im
"`Lapidarstil"' \pc{170}{kap} das zinstragende Kapital vorgestellt ist
und was zugegebenermaßen schon eine bemerkenswerte Eigenschaft wäre.
In seiner allgemeinen Form ist Kapital Geld, das über den Umweg seiner
Verausgabung als mehr Geld zu sich selbst zurückkommt. Es handelt sich
also nicht nur um eine Vermehrung, eine Steigerun, sondern obendrein
um eine Selbst-Verausgabung (?) und eine Identität durch diese
Verausgabung hindurch. Das Kapital ist Kapital durch einen zweifachen
Wechsel der Form, einmal von Geld zu Ware und einmal von Ware zurück
zu Geld: "`beide, Ware und Geld, [funktionieren] nur als verschiedne
Existenzweisen des Werts selbst [\lips] Er geht beständig aus der
einen Form in die andre über, ohne sich in dieser Bewegung zu
verlieren"' \pc{S. 168 f.}{kap}.

%"`, und verwandelt sich so in ein automatisches Subjekt"' \pc{S. 168 f.}{kap}.

%\worries{vs. S. 169, wo Marx den Wert "`Substanz"' nennt}

%Das Kapital "`ist immer sich selbst voraus, stets entgeht ein Rest der
%Zuschreibung"' \pc{125}{strauss}. Selbstüberschreitung

-- Kapital als Subjekt

Selbst angetrieben, die ständigen Verwandlungen überstehend, wird das
Kapital zum übergreifenden Subjekt seiner eigenen Verwertung. Das
Kapital ist das "`Subjekt der beschriebenen Bewegung, die es selbst
als sein eigner Verwertungsprozeß ist"' \zn{Marx}{181}{reichelt}. In
der Bewegung der Selbstverwertung nennt Marx das Kapital schließlich
"`automatisches Subjekt"' \pc{169}{kap}.

"`In der Tat aber wird der Wert hier das Subjekt eines Prozesses,
worin er unter dem beständigen Wechsel der Formen von Geld und Ware
seine Größe selbst verändert, sich als Mehrwert von sich selbst als
ursprünglichem Wert abstößt, sich selbst verwertet. Denn die Bewegung,
worin er Mehrwert zusetzt, ist seine eigne Bewegung, seine Verwertung
also Selbstverwertung"' \pc{169}{kap}.

-- Widerspruch / Selbst-Negation

Die Schwierigkeit, den Kapitabegriff zu fassen, lässt sich vielleicht
ablesen an der Vielzahl verschiedener Formulierungen, mit denen Marx
versucht, das Kapital zu charakterisieren. Nicht nur ist es
"`Subjekt"' und "`stößt sich von sich selbst ab"', es ist außerdem
"`selbst der prozessierende Widerspruch"' \pc{601}{grundr}. Am
bemerkenswertesten ist aber eine Formulierung, die Helmut Reichelt
zitiert: "`Das Kapital ist daher in jeder besonderen Phase die
Negation seiner als des Subjekts der verschiednen Wandlungen"'
\zn{Marx}{181}{reichelt}.

Diese eigentümliche Bewegung des Kapitals, die über einen Umweg zu
sich selbst zurückkommt, wird hier verstanden als Negation. Die
Identität des Kapitals lässt sich nur noch bestimmen als substanzlose
Identität in der Differenz der Steigerung und der zweifachen
Verwandlung, ja der Selbst-Negation.

% das sich selbstnegierende Subjekt

Bevor ich mich allerdings der subjekt-theoretischen Annäherung an den
Begriff des Humankapitals -- und insbesondere der Analyse der
Konsequenzen der Analyse dieses Kapital-Subjekts für die Vorstellung
eines Humankapital-Subjekts -- widme, stellt sich die Frage, wie es um
theoretische Fundierung der beiden Bestandteile dieses Begriffs
bestellt ist.

\subsubsection{Exkurs: Humankapital ein Kapital im Sinne der
Arbeitswerttheorie?}

Im Anschluss an seine Analysen der neoliberalen Diskurse führt
Foucault den Begriff des Humankapitals folgendermaßen ein. Eine
bestimmte, subjektive Werttheorie versteht alles als Kapital, was in der
Lage ist, Einkommen zu generieren. ... \worries{?}

... wobei "`Kapital"' in diesem Verständnis alles bezeichnet, "`was
auf die eine oder andere Weise eine Quelle von zukünftigem Einkommen
sein kann"' \pc{312}{gbp}. (Arbeit wird dann so definiert, "`daß es
ein zukünftiges Einkommen ermöglicht, welches der Lohn ist"'
\pc{312}{gbp}.) Das besondere an diesem Kapital "`Arbeit"' ist aber
natürlich, dass es "`praktisch untrennbar von der Person ist, die es
besitzt"' \pc{312}{gbp} -- so die Begriffsbildung \emph{Human}kapital.

---

Nach der Rekonstruktion der Marx'schen Definition des Kapitals stellt
sich die Frage, inwiefern die erläuterte neoliberale Konzeption des
Humankapitals damit in Einklang zu bringen ist. Mit dieser Frage hat
sich auch Harald Strauß in einem Vortrag beschäftigt -- und sein
Urteil fällt eindeutig aus: "`Der Arbeitswerttheorie -- ob in den
Varianten von Smith, Ricardo oder Marx -- ist der Begriff Humankapital
gänzlich inkompatibel"' \pc{124}{strauss}.

Im neoliberalen Verständnis setzt sich da Humankapital zusammen aus
angeborenen und erworbenen Bestandteilen \vgl{316}{gbp}, aus
Fähigkeite, für deren Einsatz die Arbeiterin einen Lohn erhält. Aus
mehr Humankapital, durch mehr "`Investitionen"', entspringe auch mehr
Einkommen. Diese Einschätzung beantwortet Strauß mit einer deutlichen
Verschiebung des Blickwinkels. Die Fähigkeiten einer Arbeiterin zu
arbeiten parallelisiert er mit der Kategorie der Arbeitskraft und eben
nicht mit der des Kapitals: "`Gute Bildung, Gesundheit und Manieren
können durchaus nützlich sein, doch sind sie keine Bestandteile eines
Kapitals, sie hecken kein Geld. Sie sind vielmehr, was auf dem
jeweiligen Stand der Produktivkraftentwicklung von jeder Arbeitskraft
erwartet werden darf, nichts, was die abhängig Beschäftigten je in die
Position eines Unternehmers, vulgo: Kapitalisten bringe würde"'
\pc{128}{strauss}.

Die Fähigkeiten einer Arbeiterin, von der neoliberalen Theorie
verstanden als Resultat von Investitionen ins Humankapital, schlägt
Strauß dagegen dem Gesamtpaket der Ware Arbeitskraft zu, die in der
Zirkulation für Geld gekauft wird. Die Anwendung dieser Fähigkeiten,
d.\,h. der Gebrauch der Arbeitskraft, d.\,h. die Arbeit findet
allerdings außerhalb dieser Sphäre statt. Hier, in der Produktion,
erscheint die Arbeit als Gebrauchswert der Arbeitskraft, hier erst
werden die Fähigkeiten angewandt. Damit sind sie aber Gebrauchswert
und nicht Tauschwert: "`Die Fähigkeiten eines abhängig Beschäftigten
sind gerade aufgrund ihres Gebrauchswertcharakters von der Kapitalform
ausgeschlossen"' \pc{126}{strauss}.

Zwar taucht die Arbeiterin natürlich in den Rechnungen des Kapitals
auf, nämlich als von Marx sogenanntes "`variables Kapital"'
\pc{224}{kap}. Doch geht es dabei um das Geld, das zum Einkauf ihrer
Arbeitskraft ausgegeben wurde. \worries{Stimmt das?} Zum einen funktioniert es
nicht, die ganze Arbeiterin mit dem variablen Kapital gleichzusetzen:
"`Eine Arbeiterin ist mitnichten \emph{variables Kapital}, sondern es
ist ihre Lohnsumme
%\emph{v}
$v$, die einen bestimmten Teil der Kapitalrechnung ausmacht"'
\pc{126}{strauss}. Zum anderen, so ließe sich hinzufügen, taucht auch
nicht die ganze geleistet \emph{Arbeit} als variables Kapital in der
Rechnung wieder auf. Die Einführung der Differenz von Arbeit und
Arbeitskraft ist gerade die große Entdeckung von Marx zur Erklärung
des Ursprungs von Mehrwert \vgl{xxx}{kap}.

So tauscht die Arbeiterin ihre zur Ware gewordene Arbeitskraft in der
Zirkulation gegen Geld, während ihre Arbeit in der Produktion mehr
Wert schafft, als ihre Arbeitskraft wert ist. Diese Differenz kann die
Gleichsetzung von Arbeit und Kapital nicht erklären. Zusätzlich
entgeht ihr der Formunterschied der beiden Zirkulationsbewegungen. Aus
Sicht der Kapitalistin stellt sich der Kauf der Arbeitskraft (G--W)
als erste Hälfte eines Kapital-Kreislaufs \gwg dar. Aus der anderen
Perspektive hingegen tauscht die Arbeiterin Ware gegen Geld und
vollzieht damit die erste Hälfte des Kreislaufs der einfachen
Zirkulation. Das eingenommene Geld wird wiederum zum Kauf ein anderen
Ware verausgabt. \wgw mit klarem Ende.

So erläutert Strauß sein Urteil schließlich folgendermaßen:

\begin{spacing}{1}
\begin{quote}
"`Um sich die Kapitalform zu geben, zumal die eines Humankapitals,
müsste der Einzelne sich aufspalten, in einen Teil der die Arbeit
gegen Lohn verkauft, und einen anderen Teil, der den Gebrauch von der
Arbeit macht und das Mehrprodukt in Geldform einstreicht.
Letztlich ist es die Differenz von Arbeit und Arbeitskraft, die als
Grundlage der Mehrwertabschöpfung eine Gleichsetzung von variablem
Kapital mit Humankapital unterläuft: Weil die Differenz von Arbeit und
Arbeitskraft nicht aktiv ausgenutzt werden kann von jenen, die ihre
Haut zu Markte tragen"' \pc{126}{strauss}.
\end{quote}
\end{spacing}

...

\subsection{Mittelteil, Kontext, Krise der Disziplinargesellschaft,
Subjekt}

Es soll im weiteren Verlauf dieser Arbeit allerdings nicht um die
Diskussion verschiedener Werttheorien oder unterschiedlicher
Kapital-Definitionen gehen. Abgesehen davon \worries{?}, ob das
Humankapital "`wirklich"' Kapital  im Sinne der einen oder anderen
Definition ist, geht es mir um die Frage, welche Auswirkungen es für
Subjekte hat, wenn sie sich zu sich selbst wie Kapital bzw. wie zu
Kapital verhalten, wenn sie also in den endlos gesteigerten Kreislauf
der Verwertung eintreten. Bloß dass sich jetzt nicht eine Wertsumme
verwerten soll, sondern das Subjekt selbst.

Im Aphorismus "`Novissimum Organum"' in der Minima Moralia formuliert
Adorno die Aufgabe, aus einer Analyse des gesellschaftlichen
Zusammenhangs und konkreter, des spezifischen Zusammenhangs von
Individuen und Produktionsprozess zu einem bestimmten Zeitpunkt, "`die
innere Komposition des Individuums an sich, nicht bloß dessen
gesellschaftliche Rolle [\lips] abzuleiten"' \pc{261}{min}. Die Frage
ist also, was es für die "`innere Komposition"' eines Subjekts
bedeutet, die Form eines Humankapitals anzunehmen/annehmen zu müssen.

Denn es geht natürlich nicht um die individuelle Entscheidung der
einen oder des anderen, sich selbst humankapitalistisch zu verwerten
oder eben auch nicht und stattdessen vielleicht lieber klassisch
doppelt-freie Lohnarbeiterin zu bleiben. Fragen der Subjektform
gehören nicht (oder zumindest nicht vorrangig) in den Bereich
individueller Entscheidungen. Deshalb besitzt das Humankapital auch
eine historische Spezifik. Foucault zufolge taucht der Begriff zum
ersten Mal in den ... Jahren \worries{?} auf. Mehr noch scheint er
aber Teil der Strategien zu sein, die Verwertungskrise der 70er Jahre
zu überwinden \worries{?} \vgl{xx}{gbp}. Er gehört zum Aufkommen des
Neoliberalismus, aber eben nicht einfach in eine als wirkungslos
verstandene Gedankenwelt, sondern an den ganz materiellen Übergang von
einem fordistischen zu einem postfordistischen Produktionsregime und
einer Krise der Disziplinargesellschaft, die diesen Übergang
begleitet. Diese Diagnosen scheinen mir den Rahmen einer solchen
\worries{?} Untersuchung abzugeben.

Foucault hat davon gesprochen, wie die Neoliberalen versuchen, die
ökonomische Analyse auf Felder auszuweiten, die bisher nicht als
ökonomisch galten \vgl{305}{gbp}, also gewissermaßen versuchen, "`das
ökonomische Modell in großem Maßstab zur Anwendung zu bringen"'
\pc{334}{gbp}. Aus der Ökonomie werde so nicht nur ein "`Modell für
die sozialen Beziehungen"' \pc{334}{gbp}, sondern ein "`Modell der
Existenz selbst"' \pc{334}{gbp}. Diese Feststellung ist für diese
Arbeit natürlich von besonderer Bedeutung. Wir aus dem ökonomischen
Modell eine bestimmte "`Form der Beziehung des Individuums zu sich
selbst"' \pc{334}{gbp} (und nicht nur zu sich selbst, sondern auch
"`zur Zeit, zu seiner Umgebung, zur Zukunft, zur Gruppe, zur Familie"'
\pc{334}{gbp}), dann ist die Subjektform des Humankapital als Resultat
dieser wirkmächtig gewordenen Diskurse zu analysieren: Humankapital
als ökonomisches Subjekt, das in der Verwertungslogik des Kapitals
steht -- welche Gestalt das auch annehmen mag, wenn nicht der Wert
sondern das Subjekt sich bewegt.

Nicht nur bei Foucault geht die Verallgemeinerung der Ökonomie mit
einer besonderen Bedeutung des Begriffs oder der Figur oder der Form,
des Diskurses des "`Unternehmens"' einher. Auch in Gilles Deleuze'
kleinem Text \citetitle{ps} \pc{}{ps} ist "`das Unternehmen"' als
Nachfolger der Fabrik zentral. Und man denke nur an die ganzen
Begriffsbildungen im Anschluss an Foucault, wie
"`Arbeitskraftunternehmer"' oder "`unternehmerisches Selbst"'
\worries{?}.

Im Zuge der Verallgemeinerung der Ökonomie als Modell für alles,
inbesondere für die Existenz und das Selbstverhältnis der Individuen
wird aus dem ökonomischen Menschen, dem \hoe, auch ein Unternehmer:
"`und zwar ein Unternehmer seiner selbst"' \pc{314}{gbp}. Deleuze
weitet das Urteil noch aus: inzwischen sei "`an die Stelle der Fabrik
das Unternehmen"' getreten \pc{256}{ps}. Überhaupt seien "`Familie,
Schule, Armee, Fabrike [\lips] keine unterschiedlichen analogen
Milieus mehr [\lips{} sondern] chiffrierte, deformierbare und
transformierbare Figuren ein und desselben Unternehmens, das [-- ganz
im Einklang mit Foucaults Wort vom Selbst-Unternehmer --] nur noch
Geschäftsführer kennt"' \pc{260}{ps}. Die neoliberale
Wirtschaftsanalyse setze als ihr "`Grundelement"' \pc{313}{gbp} nicht
mehr das Individuum, sondern das Unternehmen: "`eine Gesellschaft aus
Unternehmenseinheiten"' \pc{313}{gbp}. Folgerichtig versteht sich das
arbeitende Subjekt nicht bloß als Unternehmen, sondern -- was für die
subjekttheoretischen Implikationen sicherlich noch spannender ist --
gleich als ganzes Unternehmen: "`so daß der Arbeiter selbst sich als
eine Art von Unternehmen erscheint"' \pc{313}{gbp}.

--

Die Untersuchung der Ablösung der Fabrik durch das Unternehmen gehören
in den allgemeineren Kontext einer Diagnose der "`Krise aller
Einschließungsmilieus"' \pc{255}{ps}, d.\,h. der ganzen
Disziplinargesellschaft, die Deleuze im \citetitle{ps} formuliert. Und
Deleuze hat auch schon einen Namen für das, was danach kommt, für die
post-disziplinäre Gesellschaft, nämlich die titelgebende
"`Kontrollgesellschaft"', die nach den "`kurzfristig[en] und auf
shcnellen Umsatz gerichtet[en], aber auch kontinuierlich[en] und
unbegrenzt[en]"' \pc{260}{ps}, ja "`ultra-schnellen Kontrollformen"'
\pc{255}{ps} benannt ist.

Bevor man allerdings diese Unterscheidungen und vielleicht auch die
ganze Krisendiagnose nur der Eigentümlichkeit von Deleuze' Stil
zuschreibt, sollte bemerkt werden, dass auch Foucault in \cite{gbp}
eine ganz ähnliche Diagnose vornimmt. Seine Untersuchungen der
"`Geschichte der Gouvernementalität"', wie \cite{stb} und \cite{gbp}
im Untertitel heißen, stehen ebenfalls im Zeichen einer Krise der
Disziplinargesellschaft, oder zumindest eines Abrückens vom Ziel
"`einer erschöpfend disziplinarischen Gesellschaft"' \pc{359}{gbp}.
Denn -- immer noch im Zuge einer Ausweitung und Verallgemeinerung des
ökonomischen Modells auf alle gesellschaftlichen Bereiche -- es setzen
sich nicht nur andere Analyseraster durch, sondern auch damit
zusammenhängende Regierungsverfahren und -techniken. In dieser
veränderten Konzeption und angesichts veränderter Möglichkeiten der
Regierung, ja vielleicht der Deleuze'schen Kontrolle, hat die
Gesellschaft "`kein unbegrenztes Bedürfnis nach Konformität. Die
Gesellschaft braucht sich keineswegs einem erschöpfenden
Disziplinarsystem zu unterwerfen"' \pc{354}{gbp}. So nimmt sich das
ökonomische Denken auch des paradigmatischen Falls der Einschließung,
nämlich der Verbrechen und Strafen an. Neoliberal reformuliert "`zielt
die richtige Strafpolitik keineswegs auf die Auslöschung des
Verbrechens, sondern auf ein Gleichgewicht zwischen den Kurven des
Angebots an Verbrechen und der negativen Nachfrage"' \pc{354}{gbp}.

Es lässt sich hier also auch bei Foucault eine deutliche Verschiebung
feststellen. Etwas, das Foucault "`Umwelttechnologie"' nennt
\pc{359}{gbp} ersetzt die Vorstellung einer "`Gesellschaft in der ein
Mechanismus der allgemeinen Normalisierung und des Ausschlusses des
Nicht-Normalisierbaren erforderlich wäre"' \pc{359}{gbp}. Bildlich
gesprochen bedeutet das, dass es in dieser anderen Art des Regierens
"`keine Einflußnahme auf die Spieler des Spiels, sondern auf die
Spielregeln geben würde"' \pc{359}{gbp}. Das, was Foucault am Endes
Manuskripts seiner zehnten Vorlesung "`Environmentalität"' \pc{Fn., S.
361}{gbp}, wolle die Individuen also "`nicht innerlich unterwerfen"'
\pc{359}{gbp}, sondern ziele auf die Manipulation der Umwelt.

In den Ergebnissen, die Foucault feststellt, lässt sich schließlich
eine erstaunliche Parallelität zu Deleuze' Entwurf der
Kontrollgesellschaft finden. Foucault charakterisiert die Gesellschaft
des environmentalen Regierens als eine, "`in der es eine Optimierung
der Systeme von Unterschieden gäbe, in der man Schwankungsprozessen
freien Raum zugestehen würde, in der es eine Toleranz gäbe, die man
den Individuen und den Praktiken von Minderheiten zugesteht"'
\pc{359}{gbp}. Statt Ausschluss also gewissermaßen Regierung der
Differenzen. Ich denke, dass sich in diesem Sinne auch Deleuze' Rede
von der "`\emph{Modulation}"' \pc{256}{ps}, die an die Stelle der
verschiedenen "`\emph{Formen}"' \pc{256}{ps} der Einschließung
getreten ist, deuten lässt. Es gibt keine klar voneinander getrennten
Bereiche mehr, keine "`Milieus"', denen man -- und sei es durch den
Wechsel in eine andere Einschließung, Schule, Fabrik, Krankenhaus --
entkommen könnte. Stattdessen: "`untrennbare Variation"' \pc{256}{ps}
der Kontrollmechanismen, solange man nur "`in jedem Moment die
Position eines [und jedes] Elements in einem offenen Milieu"'
\pc{261}{ps} kennt.

--

Diese Verschiebungen auf Seiten der Regierung lassen sich an dieser
Stelle allerdings nur anreißen. Der Schwerpunkt dieser Arbeit liegt
auf ihrem Gegenstück: auf dem Subjekt, das auf diese Weise regierbar
ist.

Auch Foucault sucht nach einem bestimmten Subjekt "`als Partner, als
Gegenüber, als Basiselement der neuen gouvernementalen Vernunft"'
\pc{372}{gbp} (auch wenn er hier über das 18. Jahrhundert spricht). Er
findet es im \hoe \vgl{Vorlesung 11, S. 367-398}{gbp}.

An dieser Stelle muss man wohl Foucault mit Foucault selbst
widersprechen. Noch im letzten, nicht mehr vorgetragenen Teil des
Manuskripts von Vorlesung 10 von \cite{gbp} hatte er die Frage
aufgeworfen, ob die Verschiebung zur Environmentalität, d.\,h. einer
Technik des Regierens, die auf Regel- und Unwelt-Manipulation setze,
angesichts derer die Subjekte dann ihr Verhalten anpassen müssten,
statt sie direkt zu unterwerfen, bedeute "`daß man es mit natürlich
Subjekten zu tun hat"' \pc{Fn., S. 361}{gbp}. Im Grunde gibt sich
Foucault mit der Analyse des \hoe selbst die Antwort -- einmal davon
abgesehen, was man sich im Kontext seiner Untersuchungen überhaupt
unter einem \emph{natürlichen} Subjekt vorzustellen hätte.

Environmentale Regierungstechniken mögen zwar nicht auf unmittelbare
Unterwerfung, sondern auf die Manipulation der Umwelt, der
"`Gegenbenheiten des Spiels"' \pc{Fn., S. 360}{gbp} zielen. Es scheint
mir aber unpassend \worries{?} davon auf eine gewissermaßen
unterwerfungsfreie Regierung zu schließen, angesichts derer
"`natürliche Subjekte"' selbst ihr Verhalten anpassen.
Konsequenterweise scheint mir der Schritt stattdessen von Unterwerfung
zur Selbst-Unterwerfung zu gehen. Und für das Humankapital als Subjekt
heißt das: Selbst-Verwertung als Selbst-Unterwerfung.

Subjekt kapitalförmig: alles paletti für die Regierungsbemühungen des
Kapitals

Liberalismus als Seins- und Denkweise, Beziehung zwischen Regierenden
und Regierten \vgl{305}{gbp}.

\subsubsection{Reformulierung für das Subjekt}

Es soll im weiteren Verlauf dieser Arbeit allerdings nicht um die
Diskussion der Probleme einer subjektiven Werttheorie, etwa im
Vergleich zu einer "`objektiven"' Arbeitswertheorie, gehen, oder
darum, ob das Humankapital nun "`wirklich"' ein Kapital ist, sondern
...

Es geht nicht darum, ob es "`wirklich"' ein Kapital ist, sondern um
die spezifische Subjektivierungsweise, um das Selbstverhältnis, das in
der Vorstellung der Menschen als Humankapital ausgedrückt ist.

Selbstverwertung des Werts und Selbstverwertung des Subjekts

Wenn Kapital als Subjekt verstanden, dann:
Erschließung --> Selbsterschließung
Steigerung --> Selbststeigerung usw.

Selbstverwertung des Werts --> Selbstverwertung des Selbst

Adorno-Zitat ("`innere Komposition ableiten"')

% {{{ Subjekt

\subsection{Subjekt}

\subsubsection{Steigerungslogik, Hdgg. usw.}

"`Selbstübermächtigung"'

"`Wille will sich selbst"' --> der Wert verwertet sich selbst

\subsubsection{Diverses}

\subsubsection{Neubestimmung des Subjekts}

Reformulierung von Reichelt:

"`am Ende der Darstellung wird sich zeigen, daß es das Kapital selbst
ist, das uns in verschiedenen Formen begegnet, die sich alle als
Momente seiner selbst erweisen"' \pc{181}{reichelt}.

Besser \worries{?}: "`... daß es das Kapital selbst ist, das
\emph{sich} in verschiedenen Formen begegnet"'

% }}}

% Schizophrenie {{{

\subsection{Schizophrenes Subjekt}

Nachdem dieser Kontext für des Humankapitals geklärt ist, geht es
jetzt erneut an die Aufgabe, die Adorno formuliert hat. Die "`innere
Komposition"' \pc{261}{min} des Subjekt ist abzuleiten.

Ausgehend von einer Nebenbemerkung von Strauß über das Humankapital
wird meine These sein, dass das humankapitalistisch-post-disziplinäre
Subjekt schizophren ist. Dabei werde ich mich vor allem auf Deleuze
und Guattaris \citetitle{ao} beziehen. Auf die allgemeine Diskussion
der Schizophrenie und des Wortpaares, das auch den Untertitle des
\citetitle{ao} bildet, nämlich \emph{Kapitalismus und Schizophrenie},
geht es darum, die schizophrenen Strukturen auch in dem spezifischen
Selbstverhältnis wiederzufinden, das durch die Konzeption des
Humankapitals impliziert ist.

In seinem Vortrag über das Humankapital hatte Harald Strauß gesagt:
"`Um sich die Kapitalform zu geben, zumal die eines Humankapitals,
müsste der Einzelne sich aufspalten, in einen Teil der die Arbeit
gegen Lohn verkauft, und einen anderen Teil, der den Gebrauch von der
Arbeit macht und das Mehrprodukt in Geldform einstreicht"' \pc{S.
126}{strauss}.\footnote{Wobei es korrekter Weise heißen müsste, dass
ein Teil die Arbeits\emph{kraft} verkauft, die der andere Teil kauft,
und nicht die \emph{Arbeit}, wie es im Zitat heißt.} Meine These nimmt
Anstoß an diesem Gedanken des gespaltenen Subjekts, wobei ich glaube,
dass sich auch andere Spaltungen und Differenzen ausmachen lassen, als
die zwischen Arbeitskraft verkaufendem und Arbeitskraft kaufendem
Teil, ja dass das Kapital selbst und dementsprechend auf ein
kapitalförmiges Selbst schizophrene Strukturen aufweist.

%\subsubsection{Schizophrenie bei Deleuze}
\subsubsection{Schizophrenie und Deleuze}

Auch wenn die Rede von einem gespaltenen Individuum den Aufhänger für
diesen Teil der Arbeit bildet, so ist doch die Diagnose der
Schizophrenie, die Deleuze und Guattari formulieren, nicht einfach mit
der landläufigen Bedeutung als "`gespaltene Persönlichkeit"' zu
übersetzen. Dahinter steckt vielmehr ein medizinisch, psychologisch
und philosophisch komplexes Konzept.

Schizophrenie gilt als die Krankheit ohne einheitliches Krankheitsbild
\worries{vgl XYZ!}. Zur Abgrenzung identifizert Deleuze "`zwei Systeme
von Verrückten"' \pc{12}{zweisys}, von denen die Paranoia das eine und
die Schizophrenie das andere bildet \vgl{S. 14 f.}{zweisys}. So hat
der Wahn \worries{?} zwei Formen: die "`paranoide Form und [die]
wunderwirkende oder phantastische Form der Schizophrenie"' \pc{S. 21
f.}{schizg}.

-- Paranoia

-- Schizophrenie

Das entscheidende Charakteristikum der Schizophrenie ist dagegen
sicherlich die Diskontinuität.

"`Als Eugen Bleuler im Jahre 1911 den Terminus Schizophrenie erfindet,
weist er nachdrücklich auf eine Zersplitterung oder funktionale
Dislokation der Assoziationen hin, die den fehlenden Zusammenhang zur
Hauptstörung macht"' \pc{23}{schizg}, deren "`Kehrseite"' eine
"`Zersetzung der Person"' und eine "`Abspaltung von der Realität
[sind], die einem starren und sich selbst verschlossenen Innenleben
eine art Übergewicht oder Autonomie verleihen"' \pc{23}{schizg} --
das, was üblicherweise Autismus genannt wird.

"`Einheit der Schizophrenie"' ließe sich wohl nur "`in der Gesamtheit
einer gestörten Persönlichkeit"' denken, oder "`in den psychotischen
Formen des \glq Auf-der-Welt-Seins\grq\,"' \pc{24}{schizg}.

"`Spaltung ist ein schlechtes Wort zur Bezeichnung des Zustands der
Elemente, die in diese speziellen Maschinen eingehen, die positiv
bestimmbaren schizophrenen Maschinen -- hier haben wir auf die
maschinelle Rolle des fehlenden Zusammenhangs hingewiesen"'
\pc{27}{schizg}.

Die Schizophrenie ist die "`Herstellung
einer nicht-lokalisierbaren Verbindung"' \pc{19}{schizg} zwischen
heterogenen Elementen, so dass sie, "`\emph{gerade weil sie keine
Beziehung zueinander haben}, untereinander in Beziehung treten"'
\pc{19}{schizg}.

Die "`beiden Pole der Schizophrenie"' \pc{21}{schizg} sind die
katatonischen Krampfzustände und die wahnhafte Aktivität, "`Katatonie
des organlosen Körpers, anorganische Tätigkeit der
Organmaschinen"'\pc{21}{schizg}, zwischen denen vielfache Übergänge
und Wechsel stattfinden.

Ende: Explizit positiv: "`Dennoch besteht die Schwierigkeit darin, der
Schizophrenie in ihrer Positivität und als Positivität Rechnung zu
tragen"' \pc{24}{schizg}, und auf die "`Merkmale des Defizits"', der "`
Zerstörung"', der "`Lücken und Spaltungen"' \pc{24}{schizg} zu
reduzieren.

Schizophrenie als "`\emph{Prozeß}"' \pc{27}{schizg}, als
"`schizophrene Reise"' \pc{22}{schizg} nicht Mangel oder Zerstörung

"`Ganz anders [als die anderen, nicht als ich] verstehen Karl Jaspers
und heute Ronald D. Laing den prallen Begriff \glq Prozeß\grq: Als
Bruch, Einbruch, Durchbruch, der die Kontinuität einer Persönlichkeit
unterbricht und sie auf eine Art Reise schickt, durch ein intesives
und erschreckendes \glq Mehr an Realität\grq{} hindurch, gemäß
Fluchtlinien, in denen Natur und Geschichte, Organismus und Geist sich
verfangen"' \pc{28}{schizg}. \worries{aber bei denen doch auch ein Problem?}

"`Genau das spielt sich zwischen den schizophrenen Maschinenorganen,
den organlosen Körpern und den Intensitätsströmen auf diesem Körper
ab, was eine ausgedehnte Verzweigung von Maschinen und eine gewaltige
Drift der Geschichte verursacht"' \pc{28}{schizg}.

"`Wenn die Schizophrenie als die Krankheit der heutigen Zeit
erscheint, dann nicht aufgrund von Allgemeinheiten, die unsere
Lebensweise betreffen, sondern in bezug auf äußerst präzise
Mechanismen ökonomischer, sozialer und politischer Natur"' \pc{28}{schizg}.

Und diese Mechanismen gilt es ausfindig zu machen/zu benennen

Schiz. weil im Kap. Decodierung und Deterritorialisierung. "`Im
Gegensatz zum Paranoiker, dessen Wahn darin besteht, Codes
wiederherzustellen, Territorialitäten aufs neue zu erfinden, hört der
Schizophrene nicht auf, immer weiter zu gehen in der Bewegung, sich
selbst zu decodieren, sich zu deterritorialisieren (Durchbruch, Reise
oder schizophrener Prozeß)"'\pc{28}{schizg}.

\subsubsection{Schizophrenie und Kapitalismus}

So wird aus Connecticut "`Connect -- I -- cut"' \pc{48}{ao}

Kapital auch gespalten wegen Gott-Vater und Gott-Sohn, Indentität
hinterher, erst nach der Bewegung, aber da natürlich auch Differenz.

"`Er [der Wert] unterscheidet sich als ursprünglicher Wert von sich
selbst als Mehrwert, als Gott Vater von sich selbst als Gott Sohn, und
beide sind vom selben Alter und bilden in der Tat nur eine Person,
denn nur durch den Mehrwert von 10 Pfd. St. werden die vorgeschossenen
100 Pfd. St. Kapital, und sobald sie dies geworden, sobald der Sohn
und durch den Sohn der Vater erzeugt, verschwindet ihr Unterschied
wieder und sind beide Eins, 110 Pfd. St."' \pc{S. 169 f.}{kap}.

DIESES GANZE SUBJEKT-DIFFERENZ-DING?!

Kontinuität im Bruch, Kontinuität durch den Einschnitte hindurch, ja
sogar das \emph{I}, das Ich, das Subjekt, das "`ich"' sagt, taucht an
der Stelle zwischen Verbindung und Schnitt auf, befindet sich
gewissermaßen mitten im Schnitt, oder ist die Differenz zwischen
Kontinuität/Identität und Differenz, Verbindung und Schnitt. Das
Subjekt des Kapitals. Wir haben \gwg.

Verwertung des Werts als schizophrene Reise

\gwg verliert "`Sinn und Verstand"' \pc{166}{kap}

"`Die rastlose Vermehrung des Werts, die der Schatzbildner anstrebt,
indem er das Geld vor der Zirkulation zu retten sucht, erreicht der
klügere Kapitalist, indem er es stets von neuem der Zirkulation
preisgibt"' \pc{168}{kap}.

"`In der Tat aber wird der Wert hier das Subjekt eines Prozesses,
worin er unter dem beständigen Wechsel der Formen von Geld und Ware
seine Größe selbst verändert, sich als Mehrwert von sich selbst als
ursprünglichem Wert abstößt, sich selbst verwertet. Denn die Bewegung,
worin er Mehrwert zusetzt, ist seine eigne Bewegung, seine Verwertung
also Selbstverwertung"' \pc{169}{kap}.

"`Als das übergreifende Subjekt eines solches Prozesses, worin er
Geldform und Warenform bald annimmt, bald abstreift, sich aber in
diesem Wechsel erhält und ausreckt, bedarf der Wert vor allem einer
selbständigen Form, wodurch seine Identität mit sich selbst
konstatiert wird. Und diese Form besitzt er nur im Gelde"'
\pc{169}{kap}.

\subsubsection{Schizophrenie und Humankapital}

Humankapital als automatisches Subjekt? Was ist da mit dem
Formwechsel?

"`Er [der Eigentümer der Arbeitskraft] als Person muß sich beständig
zu seiner Arbeitskraft als seinem Eigentum und daher seiner eignen
Ware verhalten"' \pc{182}{kap}.

Der Wer tritt "`sozusagen in ein Privatverhältnis zu sich selbst"'
\pc{169}{kap}

% }}}

% Aufräumen {{{
\section{Aufräumen}

Zurück zu Foucault?!

% }}}

\newpage
\nocite{*}
\printshorthands
%\newpage
\printbibliography

\end{document}
