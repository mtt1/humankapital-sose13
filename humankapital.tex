% preamble {{{
\documentclass[12pt,
               DIV13,
               paper=a4,
               twoside=false,
               onehalfspacing,
               %titlepage,
               bibliography=totoc,
               toc=graduated,
               draft,
               ]{scrartcl}

\usepackage[utf8]{inputenc}
\usepackage[T1]{fontenc}
%\usepackage{ngerman}
%\usepackage[british]{babel}
\usepackage[ngerman]{babel}
\usepackage[babel,german=quotes]{csquotes}
\usepackage{setspace}
%\usepackage{mathptmx}           % pslatex's successor
%\usepackage[scaled=.92]{helvet} % pslatex's successor
%\usepackage{courier}            % pslatex's successor
\usepackage[osf]{libertine}
\usepackage{courier}
% GEHT NICHT?:
%\usefont{T1}{fxlj}{m}{n}\selectfont % Mit Zahlen, die nach unten hängen
\usepackage{color}
\usepackage{ifpdf}
\usepackage{scrpage2}
\usepackage{xspace}

\usepackage[backend=biber,
            sortlocale=de,
            %style=authoryear,
            %citestyle=authoryear-ibid,
            citestyle=authoryear-icomp,
            bibstyle=authoryear,
            %natbib=true,
            sortlos=los,
            %autopunct=true,
            language=ngerman,
            clearlang=true,
            %babel=none,
            block=none,
            %ibidtracker=constrict, % automatically set by authoryear-icomp
            loccittracker=constrict, % no page-number after "ibidem" if the same page is cited again
            ]{biblatex}
\addbibresource{literatur.bib}

\ifpdf
 \usepackage[pdftex]{graphicx}
 \DeclareGraphicsExtensions{.pdf}
 \pdfcompresslevel=9
% \usepackage[%
%   pdftex=true,
%   backref=true,
%   linktocpage=true,
%   pdfpagemode=None
% ]{hyperref}
% \hypersetup{
%   pdftitle={},
%   pdfauthor={Matthias Rudolph},
%   pdfsubject={},
%   pdfcreator={LaTeX2e and pdfLaTeX},
%   pdfproducer={},
%   pdfkeywords={}
% }
\else
  \usepackage[dvips]{graphicx}
  \DeclareGraphicsExtensions{.eps}
\fi

%\setcounter{secnumdepth}{-1} % keine section-Nummerierung

\pagestyle{scrheadings}
%\automark[section]{subsection}

\ihead[]{}
\chead[]{TITEL} % Name der Hausarbeit
\ohead[]{Matthias Rudolph}
\ifoot[]{}
\cfoot[]{}
\ofoot[]{\thepage} % Seitenzahl

% Biblatex hacks
%
% Quick 'n' dirty
% Richtiger Artikel für "Hrsg. *vom* Institut"
\DefineBibliographyStrings{ngerman}{%
    bytranslator = {hrsg\adddotspace vom},
}

% Kein "Bd." in \volcite (trotzdem per Komma getrennt)
%\DeclareFieldFormat{volcitevolume}{#1}

% Doppelpunkt statt Punkt nach dem Label
\renewcommand{\labelnamepunct}{\addcolon\space}

% Schrägstriche zwischen mehreren AutorInnen
\renewcommand*{\multinamedelim}{\addslash}
\renewcommand*{\finalnamedelim}{\addslash}

% Alle Bibliografie-Einträge: Nachname, Vorname. Auch für Einträge
% mit mehreren AutorInnen und die Sigel-Liste.
% http://tex.stackexchange.com/questions/12806/guidelines-for-customizing-biblatex-styles
\DeclareNameAlias{sortname}{last-first} % Bibliografie
\DeclareNameAlias{default}{last-first} % Vollzitate (?)
%\DeclareNameAlias{labelname}{last-first} % alle anderen Zitate

% Nachname in Kapitälchen. Effekt aber nicht nur in der Bibliographie,
% sondern auch bei Zitaten.
%\renewcommand*{\mkbibnamelast}[1]{\textsc{#1}}

% More space between different entries in the bibliography.
% See \bibitemsep, \bibnamesep, \bibinitsep
% http://tex.stackexchange.com/questions/19105/how-can-i-put-more-space-between-bibliography-entries-biblatex
\setlength\bibnamesep{1em}

\newcommand{\freel}{\vspace{1em}}
\newcommand{\lips}{\dots\unkern}

\newcommand{\tit}[1]{\textit{#1}}
\newcommand{\tbf}[1]{\textbf{#1}}

\newcommand{\pc}[2]{\parencite[#1]{#2}}
\newcommand{\vgl}[2]{\parencite[vgl.][#1]{#2}}
\newcommand{\zn}[3]{\parencite[#1, zit. nach][#2]{#3}}

% Disable single lines at the start of a paragraph (Schusterjungen)
\clubpenalty = 10000

% Disable single lines at the end of a paragraph (Hurenkinder)
\widowpenalty = 10000
\displaywidowpenalty = 10000

% Don't insert more space at the end of a sentence than between words.
\frenchspacing

% Worries in Blau
\usepackage{ifdraft}
\newcommand{\worries}[1]{\ifdraft{\textcolor{blue}{\texttt{(#1)}}}{}}

% }}}

% Für diese Arbeit {{{

% Fürs Kapital
\newcommand{\gwg}{G--W--G'\xspace}
\newcommand{\wgw}{W--G--W\xspace}

\newcommand{\hoe}{\tit{Homo oeconomicus}\xspace}

% Footnote tricks
% default:
%\deffootnote[1em]{1.5em}{1em}{%
%    \textsuperscript{\thefootnotemark}}

%\deffootnote[1em]{1em}{1em}{%
%    \textsuperscript{\thefootnotemark}}

%\deffootnote[1em]{1.5em}{1em}{\thefootnotemark\enspace}

%\deffootnote{1em}{1em}{\thefootnotemark\enspace}

% }}}

% Titel {{{

\begin{document}
\setcounter{page}{0}

\titlehead{Goethe-Universität Frankfurt\\
Fachbereich Philosophie und Geschichtswissenschaften,\\
Institut für Philosophie\\
Prof. Dr. Christoph Menke\\
Seminar: Demokratie und Kapitalismus,
SoSe 2013\\
Modul: VM 3b}
\title{Das Subjekt des Humankapitals}
\subtitle{Zwischen Steigerungslogik und Schizophrenie}
\author{Matthias Rudolph}
\date{Vorgelegt am: \today}

\maketitle
\vfill

\noindent Matthias Rudolph\\
Frankenallee 117\\
60326 Frankfurt/M\\
Matr.-Nr.: 5273120\\
Mod. Mag. Philosophie (7. FS), NF Soziologie (3. FS) \& Politikwissenschaft (2. FS)\\ % anpassen
mttrud@gmail.com
\newpage

\tableofcontents
\newpage

%}}}

\section{Einleitung}

\section{Hauptteil}

\subsection{Humankapital}

[EINLEITUNG DES HAUPTTEILS]

In seinen Gouvernementalitätsstudien \parencites[vgl.][]{stb}{gbp},
aber auch schon im fünften Kapital von \citetitle{wzw} \pc{}{wzw}
analysiert und diagnostiziert Michel Foucault das Aufkommen eines
neuen Macht-Typs, der Bio-Macht, nach, aber ebenso  in, über und
neben, der Disziplinar-Macht. \worries{Beide auf das Leben gerichtet?
WzW, 134; und die sich folgendermaßen unterscheiden ...} Ebenso
beschäftigt ihn dabei die Frage, wie ein Subjekt beschaffen sein muss,
damit es sich auf diese Art und Weise regieren lässt. In
\tit{Vorlesung 11} von \citetitle{gbp} \vgl{367-398}{gbp} analysiert
Foucault deshalb die Figur des \hoe als eine spezifische Subjektform,
die er bis zum englischen Empirismus zurückverfolgt, deren
Wiederaufleben in ökonomischen Theorien neoliberaler Prägung ihn aber
besonders interessiert. Er findet den "`\hoe als Partner, als
Gegenüber, als Basiselement der neuen gouvernementalen Vernunft, wie
sie sich im 18. Jahrhundert ausbildet"' \pc{372}{gbp}.

In diesem Kontext betrachtet Foucault die Ausweitung ökonomischer
Analysen auf Bereiche, die zunächst nicht ökonomisch scheinen oder
zumindest bisher nicht als ökonomisch betrachtet wurden, etwa Analysen
der Kriminalität oder der Ehe \vgl{367}{gbp}. Ausgeweitet auf Analysen
der Arbeit führen diese neoliberalen Analysen auf eine Vorstellung der
Menschen als Humankapital, wobei "`Kapital"' in diesem Verständnis
alles bezeichnet, "`was auf die eine oder andere Weise eine Quelle von
zukünftigem Einkommen sein kann"' \pc{312}{gbp}. (Arbeit wird dann so
definiert, "`daß es ein zukünftiges Einkommen ermöglicht, welches der
Lohn ist"' \pc{312}{gbp}.) Das besondere an diesem Kapital "`Arbeit"'
ist aber natürlich, dass es "`praktisch untrennbar von der Person ist,
die es besitzt"' \pc{312}{gbp} -- so die Begriffsbildung
\emph{Human}kapital. Mit daran anknüpfenden Konzepten, etwa von
Investition ins (eigene) Humankapital, die dann auf eine weite Palette
menschlichen Verhaltens angewandt werden kann -- von der Zeit, die
Eltern mit ihren Kindern verbringen, über Gesundheitsvorsorge bis zur
Mobilität, etwa der Bereitschaft umzuziehen \vgl{320}{gbp} --,
versucht die neoliberale Theorie, menschliches Verhalten, das
allgemein ökonomisch als Verteilung "`knappe[r] Ressourcen auf
alternative Zwecke"' \pc{310}{gbp} verstanden wird, zu erklären.

Diese Zusammenlegung von Kapital und Mensch, oder noch spezifischer
diese In-eins-Set\-zung von Kapital und Arbeit, scheint aber bereits
auf den ersten Blick problematische theoretische \worries{und
praktische?} Folgen für das vorgestellte Subjekt zu haben. Im
folgenden werde ich deshalb die Konsequenzen untersuchen, die
Bewegungen der Selbsterschließung, der Selbstverwertung, der
Selbststeigerung, die von der Vorstellung des Humankapitals impliziert
sind, für das Subjekt haben. Zunächst geht es aber um die ökonomische
Fundierung des Begriffs des Humankapitals. Dafür Karl Marx ...

\worries{Warum Marx?}

\subsection{Kapital}

\subsubsection{Kapital bei Marx}

\gwg ist die vielbeschworene Formel des Kapitals. Sie drückt aus, wie
auf dem Umweg über eine bestimmte Ware (W) aus Geld (G) mehr Geld (G')
wird. Nach der Marx'schen Arbeitswerttheorie ist das Kapital eine
Wertsumme, wobei allein Arbeit in der Lage ist, Wert zu schaffen.
Allerdings ist das Kapital nicht einfach irgendeine Wertsumme, es ist
der Wert \emph{in Bewegung}, in eben der Bewegung, die durch die
Formel \gwg beschrieben wird. Die Bestimmung des Kapitals als diese
und in dieser Bewegung hat eine Reihe weiterer Eigenschaften (?) zur
Folge, die für den Fortgang dieser Arbeit relevant sind.

-- GWG vs WGW

Wir befinden uns an dieser Stelle in der Zirkulationssphäre, in der
Ware und Geld von Privateigentümer*innen als Äquivalente getauscht
werden. Aus der einen Sicht stellt sich dieser Tausch als Verwandlung
von Geld in Ware (G--W), aus der anderen Sicht als Verwandlung von
Ware in Geld (W--G) dar. Diese Bausteine lassen sich nun
allerdings auf zwei verschiedene Weisen verketten, zum einen in der 
Bewegung \gwg, zum anderen umgekehrt als \wgw. Der erste Fall
ließe sich als "`Kaufen, um zu verkaufen"' beschreiben, der zweite als
"`Verkaufen, um zu kaufen"'. Der entscheidene Unterschied liegt in der
damit ausgesprochenen Zielsetzung: Das Ziel von \wgw ist der Austausch
einer Ware gegen Geld, um damit eine andere Ware zu kaufen. Die
Befriedigung eines Bedürfnisses mit der zweiten Ware, also ihr Entzug
aus der Zirkulation durch den Konsum, ist das Ziel. So erhält die
Bewegung \wgw einen ihr äußerlichen Zweck, und damit ihre
Grenze und ihr Ende.

Anders bei \gwg. Ziel ist hier einzig und allein die Vermehrung des
Geldes, aber eben nicht zum Entzug aus der Zirkulation. Der Schatz ist
kein Kapital. Das Geld, das über den Umweg seiner Verausgabung als
mehr Geld zu sich selbst zurückkommt, ist die eigentümliche
Existenzweise des Kapitals. Das Kapital ist sich Selbstzweck und
besitzt damit keine äußeren Schranken und Zwecke. Die Bewegung der
Steigerung ist endlos.

-- Steigerung

Das Augenfälligste ist sicherlich zunächst, dass es einen
entscheidenden Unterschied zwischen Anfang und Ende des Zweischritts
(?) gibt, nämlich einen quantitativen, weshalb es sich nicht um einen
einfachen Kreislauf handelt. Am Ende der Bewegung steht mehr Geld als
zu Beginn, das Kapital ist definiert als \gwg{} -- und nicht als
G--W--G. \worries{sonst "`abgeschmackt"' (Marx)}.\footnote{Das Rätsel,
woher diese Differenz, d.\,h. diese Zunahme an Wert im
Zirkulationsprozess kommt, löst Marx über die Einführung der Ware
Arbeitskraft, deren spezifischer Gebrauchswert Arbeit, also
wertbildende Tätigkeit, ist \vgl{xx}{kap}. Die Betrachtung der
besonderen Ware Arbeitskraft, wie der ganzen Sphäre der Produktion,
lasse ich an dieser Stelle außen vor, weil für die Bestimmung der
Steigerungslogik des Kapitals unerheblich ist, vorher die Steigerung
kommt, solange sie nur nicht als zufällige Nebenerscheinung verstanden
wird.}

Dass am Ende eines Zyklus aus Geld mehr Geld geworden ist, bestimmt
Marx als die spezifische Steigerungslogik des Kapitals. Dabei geht es
nicht um die Einzelfälle, in denen geschickte Händlerinnen ungeschickte
Käuferinnen übers Ohr hauen, sondern die Wertzunahme, die Produktion
von Mehrwert, ohne die das Kapital kein Kapital ist.

-- endlos

-- Keine Substanz

Das Kapital existiert nur in der Bewegung. Hielte man es an, würde man
das Geld sparen oder mit ihm eine Ware kaufen, die danach außerhalb
der Zirkulation konsumiert würde, wäre das Ziel gewissermaßen verfehlt
worden. Das Geld käme nicht als G' zu seinem Ausgangspunkt zurück.

-- Form-Wechsel

"`das Übergehen aus einer Phase in die andere [\lips] Das Kapital ist
daher in jeder besonderen Phase die Negation seiner als des Subjekts
der verschiednen Wandlungen"' \zn{Marx}{181}{reichelt}.

Das Kapital "`ist immer sich selbst voraus, stets entgeht ein Rest der
Zuschreibung"' \pc{125}{strauss}.

"`prozessierender Widerspruch"'

Selbst-Negation

-- KAPITAL ALS SUBJEKT

Das Kapital lässt sich schließlich als der sich selbst verwertende
Wert bestimmen, als "`ein automatisches Subjekt"' \pc{169}{kap}.

Das zirkulierende Kapital ist das "`Subjekt der beschriebenen
Bewegung, die es selbst als sein eigner Verwertungsprozeß ist"'
\zn{Marx}{181}{reichelt}.

\subsubsection{Exkurs: Humankapital ein Kapital im Sinne der
Arbeitswerttheorie?}

In einem Vortrag aus dem Jahr ... beschäftigt sich Harald Strauß unter
Anderem mit der Frage, ob das Humankapital ein Kapital im Sinne der
Arbeitswerttheorie ist.

"`Eine Arbeiterin ist mitnichten \emph{variables Kapital}, sondern es
ist ihre Lohnsumme \emph{v}, die einen bestimmten Teil der
Kapitalrechnung ausmacht"' \pc{126}{strauss}.

"`Die Fähigkeiten eines abhängig Beschäftigten sind gerade aufgrund
ihres Gebrauchswertcharakters von der Kapitalform ausgeschlossen"'
\pc{126}{strauss}.

"`Gute Bildung, Gesundheit und Manieren können durchaus nützlich sein,
doch sind sie keine Bestandteile eines Kapitals, sie hecken kein Geld.
Sie sind vielmehr, was auf dem jeweiligen Stand der
Produktivkraftentwicklung von jeder Arbeitskraft erwartet werden darf,
nichts, was die abhängig Beschäftigten je in die Position eines
Unternehmers, vulgo: Kapitalisten bringe würde"' \pc{128}{strauss}.

"`Letztlich ist es die Differenz von Arbeit und Arbeitskraft, die als
Grundlage der Mehrwertabschöpfung eine Gleichsetzung von variablem
Kapital mit Humankapital unterläuft: Weil die Differenz von Arbeit und
Arbeitskraft nicht aktiv ausgenutzt werden kann von jenen, die ihre
Haupt zu Markte tragen"' \pc{126}{strauss}

"`Der Arbeitswerttheorie -- ob in den Varianten von Smith, Ricardo
oder Marx -- ist der Begriff Humankapital gänzlich inkompatibel"'
\pc{124}{strauss}.

"`Um sich die Kapitalform zu geben, zumal die eines Humankapitals,
müsste der Einzelne sich aufspalten, in einen Teil der die Arbeit
gegen Lohn verkauft, und einen anderen Teil, der den Gebrauch von der
ARbeit macht und das Mehrprodukt in Geldform einstreicht"'
\pc{126}{strauss}.

\subsubsection{Reformulierung für das Subjekt}

Selbstverwertung des Werts und Selbstverwertung des Subjekts

Oder Erschließung --> Selbstverwertung
Steigerung --> Selbststeigerung usw.

\subsection{Subjekt}

\subsubsection{Steigerungslogik, Hdgg. usw.}

"`Selbstübermächtigung"'

"`Wille will sich selbst"'

\subsubsection{Diverses}

\subsubsection{Neubestimmung des Subjekts}

Reformulierung von Reichelt:

"`am Ende der Darstellung wird sich zeigen, daß es das Kapital selbst
ist, das uns in verschiedenen Formen begegnet, die sich alle als
Momente seiner selbst erweisen"' \pc{181}{reichelt}.

Besser \worries{?}: "`... daß es das Kapital selbst ist, das
\emph{sich} in verschiedenen Formen begegnet"'

\subsection{Schizophrenes Subjekt}

%\subsubsection{Schizophrenie bei Deleuze}
\subsubsection{Schizophrenie und Deleuze}

\subsubsection{Schizophrenie und Kapitalismus}

So wird aus Connecticut "`Connect -- I -- cut"' \pc{48}{ao}

Kontinuität im Bruch, Kontinuität durch den Einschnitte hindurch, ja
sogar das \emph{I}, das Ich, das Subjekt, das "`ich"' sagt, taucht an
der Stelle zwischen Verbindung und Schnitt auf, befindet sich
gewissermaßen mitten im Schnitt, oder ist die Differenz zwischen
Kontinuität/Identität und Differenz, Verbindung und Schnitt. Das
Subjekt des Kapitals. Wir haben \gwg.

\subsubsection{Schizophrenie und Humankapital}

\section{Aufräumen}

\newpage
\nocite{*}
\printshorthands
%\newpage
\printbibliography

\end{document}
