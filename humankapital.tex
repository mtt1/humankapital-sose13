% preamble {{{
\documentclass[12pt,
               DIV13,
               paper=a4,
               twoside=false,
               onehalfspacing,
               %titlepage,
               bibliography=totoc,
               toc=graduated,
               draft,
               ]{scrartcl}

\usepackage[utf8]{inputenc}
\usepackage[T1]{fontenc}
%\usepackage{ngerman}
%\usepackage[british]{babel}
\usepackage[ngerman]{babel}
\usepackage[babel,german=quotes]{csquotes}
\usepackage{setspace}
%\usepackage{mathptmx}           % pslatex's successor
%\usepackage[scaled=.92]{helvet} % pslatex's successor
%\usepackage{courier}            % pslatex's successor
\usepackage[osf]{libertine}
\usepackage{courier}
% GEHT NICHT?:
%\usefont{T1}{fxlj}{m}{n}\selectfont % Mit Zahlen, die nach unten hängen
\usepackage{color}
\usepackage{ifpdf}
\usepackage{scrpage2}
\usepackage{xspace}

\usepackage[backend=biber,
            sortlocale=de,
            %style=authoryear,
            %citestyle=authoryear-ibid,
            citestyle=authoryear-icomp,
            bibstyle=authoryear,
            %natbib=true,
            sortlos=los,
            %autopunct=true,
            language=ngerman,
            clearlang=true,
            %babel=none,
            block=none,
            %ibidtracker=constrict, % automatically set by authoryear-icomp
            loccittracker=constrict, % no page-number after "ibidem" if the same page is cited again
            ]{biblatex}
\addbibresource{literatur.bib}

\ifpdf
 \usepackage[pdftex]{graphicx}
 \DeclareGraphicsExtensions{.pdf}
 \pdfcompresslevel=9
% \usepackage[%
%   pdftex=true,
%   backref=true,
%   linktocpage=true,
%   pdfpagemode=None
% ]{hyperref}
% \hypersetup{
%   pdftitle={},
%   pdfauthor={Matthias Rudolph},
%   pdfsubject={},
%   pdfcreator={LaTeX2e and pdfLaTeX},
%   pdfproducer={},
%   pdfkeywords={}
% }
\else
  \usepackage[dvips]{graphicx}
  \DeclareGraphicsExtensions{.eps}
\fi

%\setcounter{secnumdepth}{-1} % keine section-Nummerierung

\pagestyle{scrheadings}
%\automark[section]{subsection}

\ihead[]{}
\chead[]{TITEL} % Name der Hausarbeit
\ohead[]{Matthias Rudolph}
\ifoot[]{}
\cfoot[]{}
\ofoot[]{\thepage} % Seitenzahl

% Biblatex hacks
%
% Quick 'n' dirty
% Richtiger Artikel für "Hrsg. *vom* Institut"
\DefineBibliographyStrings{ngerman}{%
    bytranslator = {hrsg\adddotspace vom},
}

% Kein "Bd." in \volcite (trotzdem per Komma getrennt)
%\DeclareFieldFormat{volcitevolume}{#1}

% Doppelpunkt statt Punkt nach dem Label
\renewcommand{\labelnamepunct}{\addcolon\space}

% Schrägstriche zwischen mehreren AutorInnen
\renewcommand*{\multinamedelim}{\addslash}
\renewcommand*{\finalnamedelim}{\addslash}

% Alle Bibliografie-Einträge: Nachname, Vorname. Auch für Einträge
% mit mehreren AutorInnen und die Sigel-Liste.
% http://tex.stackexchange.com/questions/12806/guidelines-for-customizing-biblatex-styles
\DeclareNameAlias{sortname}{last-first} % Bibliografie
\DeclareNameAlias{default}{last-first} % Vollzitate (?)
%\DeclareNameAlias{labelname}{last-first} % alle anderen Zitate

% Nachname in Kapitälchen. Effekt aber nicht nur in der Bibliographie,
% sondern auch bei Zitaten.
%\renewcommand*{\mkbibnamelast}[1]{\textsc{#1}}

% More space between different entries in the bibliography.
% See \bibitemsep, \bibnamesep, \bibinitsep
% http://tex.stackexchange.com/questions/19105/how-can-i-put-more-space-between-bibliography-entries-biblatex
\setlength\bibnamesep{1em}

\newcommand{\freel}{\vspace{1em}}
\newcommand{\lips}{\dots\unkern}

\newcommand{\tit}[1]{\textit{#1}}
\newcommand{\tbf}[1]{\textbf{#1}}

\newcommand{\pc}[2]{\parencite[#1]{#2}}
\newcommand{\vgl}[2]{\parencite[vgl.][#1]{#2}}

% Disable single lines at the start of a paragraph (Schusterjungen)
\clubpenalty = 10000

% Disable single lines at the end of a paragraph (Hurenkinder)
\widowpenalty = 10000
\displaywidowpenalty = 10000

% Don't insert more space at the end of a sentence than between words.
\frenchspacing

% Worries in Blau
\usepackage{ifdraft}
\newcommand{\worries}[1]{\ifdraft{\textcolor{blue}{\texttt{(#1)}}}{}}

% }}}

% Für diese Arbeit {{{

% Fürs Kapital
\newcommand{\gwg}{G--W--G'}
\newcommand{\wgw}{W--G--W}

\newcommand{\hoe}{\tit{Homo oeconomicus}\xspace}

% }}}

% Titel {{{

\begin{document}
\setcounter{page}{0}

\titlehead{Goethe-Universität Frankfurt\\
Fachbereich Philosophie und Geschichtswissenschaften,\\
Institut für Philosophie\\
Prof. Dr. Christoph Menke\\
Seminar: Demokratie und Kapitalismus,
SoSe 2013\\
Modul: VM 3b}
\title{Das Subjekt des Humankapitals}
\subtitle{Zwischen Steigerungslogik und Schizophrenie}
\author{Matthias Rudolph}
\date{Vorgelegt am: \today}

\maketitle
\vfill

\noindent Matthias Rudolph\\
Frankenallee 117\\
60326 Frankfurt/M\\
Matr.-Nr.: 5273120\\
Mod. Mag. Philosophie (7. FS), NF Soziologie (3. FS) \& Politikwissenschaft (2. FS)\\ % anpassen
mttrud@gmail.com
\newpage

\tableofcontents
\newpage

%}}}

\section{Einleitung}

\section{Hauptteil}

\subsection{Humankapital}

[EINLEITUNG ZUM HAUPTTEIL]

In seinen Gouvernementalitätsstudien
\parencites[vgl.][]{stb}{gbp}, aber auch schon im fünften Kapital von
\citetitle{wzw} \pc{}{wzw} analysiert und diagnostiziert Michel
Foucault das Aufkommen eines neuen Macht-Typs, der Bio-Macht, nach,
aber ebenso  in, über und neben, der Disziplinar-Macht. \worries{Beide
auf das Leben gerichtet? WzW, 134; und die sich folgendermaßen
unterscheiden ...} Ebenso beschäftigt ihn dabei die Frage, wie ein
Subjekt beschaffen sein muss, damit es sich auf diese Art und Weise
regieren lässt. In \tit{Vorlesung 11} von \citetitle{gbp}
\vgl{367-398}{gbp} analysiert Foucault deshalb die Figur des \hoe als
eine spezifische Subjektform, die er bis zum englischen Empirismus
zurückverfolgt, deren Wiederaufleben in ökonomischen Theorien
neoliberaler Prägung ihn aber besonders interessiert. Er findet "`\hoe
als Partner, als Gegenüber, als Basiselement der neuen
gouvernementalen Vernunft, wie sie sich im 18. Jahrhundert ausbildet"'
\pc{372}{gbp}.

In diesem Kontext betrachtet Foucault die Ausweitung ökonomischer
Analysen auf Bereiche, die zunächst nicht ökonomisch scheinen oder
zumindest bisher nicht als ökonomisch betrachtet wurden, etwa Analysen
der Kriminalität oder der Ehe \vgl{367}{gbp}. Ausgeweitet auf Analysen
der Arbeit führen diese neoliberalen Analysen auf eine Vorstellung der
Menschen als Humankapital, wobei "`Kapital"' in diesem Verständnis
alles bezeichnet, "`was auf die eine oder andere Weise eine Quelle von
zukünftigem Einkommen sein kann"' \pc{312}{gbp}. (Arbeit wird dann so
definiert, "`daß es ein zukünftiges Einkommen ermöglicht, welches der
Lohn ist"' \pc{312}{gbp}.) Das besondere an diesem Kapital "`Arbeit"'
ist aber natürlich, dass es "`praktisch untrennbar von der Person ist,
die es besitzt"' \pc{312}{gbp} -- so die Begriffsbildung
\tit{Human}kapital. Mit daran anknüpfenden Konzepten, etwa von
Investition ins (eigene) Humankapital, die dann auf eine weite Palette
menschlichen Verhaltens angewandt werden kann -- von der Zeit, die
Eltern mit ihren Kindern verbringen, über Gesundheitsvorsorge bis zur
Mobilität, etwa der Bereitschaft umzuziehen \vgl{320}{gbp} --,
versucht die neoliberale Theorie, menschliches Verhalten, das
allgemein ökonomisch als Verteilung "`knappe[r] Ressourcen auf
alternative Zwecke"' \pc{310}{gbp} verstanden wird, zu erklären.

Diese Zusammenlegung von Kapital und Mensch, oder noch spezifischer
diese In-eins-Set\-zung von Kapital und Arbeit, scheint aber bereits auf
den ersten Blick theoretische problematische Folgen für das
vorgestellte Subjekt zu haben. Im folgenden werde ich deshalb die
Konsequenzen untersuchen, die Bewegungen der Selbsterschließung, der
Selbstverwertung, der Selbststeigerung, die von der Vorstellung des
Humankapitals impliziert sind, für das Subjekt haben. Zunächst geht es
aber um die ökonomische Fundierung des Begriffs des
Humankapitals. Dafür Marx ...

\worries{Warum Marx?}

\subsection{Kapital}

\subsubsection{Kapital bei Marx}

\subsubsection{Exkurs: Humankapital ein Kapital im Sinne der
Arbeitswerttheorie?}

\subsubsection{Reformulierung für das Subjekt}

Selbstverwertung des Werts und Selbstverwertung des Subjekts

Oder Erschließung --> Selbstverwertung
Steigerung --> Selbststeigerung usw.

\subsection{Subjekt}

\subsubsection{Steigerungslogik, Hdgg. usw.}

\subsubsection{Diverses}

\subsubsection{Neubestimmung des Subjekts}

\subsection{Schizophrenes Subjekt}

\subsubsection{Schizophrenie bei Deleuze}

\subsubsection{Schizophrenie und Kapitalismus}

\subsubsection{Schizophrenie und Humankapital}

\section{Aufräumen}

\newpage
\nocite{*}
\printshorthands
%\newpage
\printbibliography

\end{document}
